\documentclass[a4paper,BCOR=10mm]{report}
\usepackage[utf8]{inputenc}
\usepackage[english]{babel}
\usepackage[T1]{fontenc}

\usepackage{amsmath}
\usepackage{amsfonts}
\usepackage{amssymb}

\usepackage[toc,page]{appendix}

\usepackage{caption}
\usepackage{color}
\usepackage{natbib}
\usepackage{algorithm,algorithmic}
\usepackage{bbm}
\usepackage{cellspace}

\usepackage{multirow}
\usepackage{hyperref}

\usepackage{tabularx}

\usepackage{graphicx}
\usepackage{csquotes}
\usepackage[begintext=``, endtext='']{quoting}
\SetBlockEnvironment{quoting}
\SetBlockThreshold{1} 

%\renewcommand{\baselinestretch}{1.5}

\newcounter{quotecount}[chapter]
\newcommand{\countquote}[2]{\vspace{1cm}\refstepcounter{quotecount}%
     (\thechapter.\arabic{quotecount}) \hspace*{1cm} \parbox{9cm}{#1\footnotemark}\footnotetext{#2} \\[1cm]}
\newcommand{\countquotenosource}[1]{\vspace{1cm}\refstepcounter{quotecount}%
     (\thechapter.\arabic{quotecount}) \hspace*{1cm} \parbox{9cm}{#1}\\[1cm]}


\renewcommand{\mkcitation}[1]{#1}
\renewcommand{\mkblockquote}[4]{\enquote{#1}#2\ifterm{\relax}{#3}#4}

\DeclareMathOperator*{\argmax}{arg\,max}
\DeclareMathOperator*{\argmin}{arg\,min}


\newtheorem{lemma}{Lemma}
\numberwithin{lemma}{chapter}

\newtheorem{definition}{Defintion}
\numberwithin{definition}{chapter}

%\newtheorem{lemma}[theorem]{Lemma}
%\newtheorem{definition}[theorem]{Definition}


%\renewcommand{\baselinestretch}{1.5}


\begin{document}

\begin{titlepage}
\pagenumbering{Roman} 
\begin{center}
 
\Large\textbf{\\
University of Heidelberg}
\vspace{2cm}

\Huge\textbf{
Abstractive Timeline Summarization
}

\vspace{10cm}

\normalsize
Master's Thesis in Computational Linguistics\\
submitted by\\
\vspace{0.5cm}
\Large\textbf{Julius Steen}\\
\normalsize
\vspace{0.5cm}
born in Stadthagen (Germany)\\
\normalsize
Matr-Nr.: 3219620 \\
\vspace{0.5cm}
\Large\textbf{May 2018} \\
\normalsize

\newpage

\normalsize
This Master's Thesis has been carried out by Julius Steen at the\\
Institut für Computerlinguistik in Heidelberg\\
under the supervision of\\
Dr. Katja Markert
\vfill
\end{center}

\newpage
\noindent
\large\textbf{Abstract}\\
\\

Timeline summarization (TLS) is the task of taking a set of documents related to some large event, for example a military conflict or natural disaster, to identify and summarize the most important dates within that larger event.
All prior work in the area of TLS, that we are aware of, has thus far focused on directly extracting interesting sentences from the original documents, an approach that is called extractive summarization. A more human-like approach to summarization is abstractive summarization which generates novel sentences to form a summarization. While many abstractive systems have been proposed for generic summarization tasks, we are not aware of any that have been used for TLS.

In this thesis we present an abstractive approach to timeline summarization that builds on existing systems for abstractive multi-document summarization (MDS). We adapt the system to better handle and take advantage of the special properties of TLS. We then evaluate this modified system and compare it to the original, unmodified system on two large TLS corpora.
We also conduct an annotation study to determine the lingustic quality of the generated summaries.

We find that the original MDS algorithm has poor performance on TLS corpora, but when adapted to TLS it outperforms an extractive baseline.
An evaluation of oracle summaries supports the idea that creating abstractive summarization as opposed to purley extractive summarizations can improve the potential performance of TLS systems.
While a manual evaluation shows that there are still significant difficulties with the grammaticality of system output, these results are encouraging for further investigation of abstractive TLS.

\newpage
\noindent
\large\textbf{Zusammenfassung}\\
\\

Timeline summarization (TLS), zu deutsch am ehesten Zeitstrahlzusammenfassung, beschreibt das Identifizieren und Zusammenfassen wichtiger Daten innerhalb großer Dokumentensammlungen über lange andauernde Ereignisse, wie beispielsweise Bürgerkriege oder Naturkatastrophen.
Alle uns bekannten Systeme für TLS setzen dafür auf sogenannte extraktive Zusammenfassung. Extraktive Zusammenfassung identifiziert wichtige Sätze in den Eingabedokumenten und kopiert sie direkt in die Zusammenfassung. Das Gegenstück zur extraktiven Zusammenfassung, die abstraktive Zusammenfassung, hingegen erstellt eine Zusammenfassung basierend auf den Inhalten der Dokumente aus neu generierten Sätzen. Sie ist damit dem gängigen Verständnis einer Textzusammenfassung deutlich näher.

In dieser Arbeit präsentieren wir einen abstraktiven Ansatz zur Timeline Summarization, der auf einem abstraktiven System für Multi-Document-Summarization (MDS) basiert. Wir stimmen dieses System auf die Besonderheiten und Anforderungen von TLS ab und evaluieren sowohl die ursprüngliche, unmodifizierte Variante, als auch die an TLS angepasste Version auf zwei großen TLS-spezifischen Corpora.
Wir evaluieren außerdem die Grammatikalität der ge~ner~ier~ten Sätze.

Die Ergebnisse zeigen, dass obwohl der originale MDS Algorithmus schlechte Ergebnisse liefert, die angepasste Variante bessere Ergebnisse erzielt, als ein extraktives Vergleichsssystem.
Eine Evaluation von Orakel-Zusammenfassungen legt weiterhin nahe, dass die Verwendung von abstraktiven Zusammenfassungen deutlich größeres Potenzial hat, als rein extraktive Systeme.
Obwohl die Gram~ma~ti~ka~li~täts~e~va~lu~a~tion signifikante Probleme in der Qualität der Systemausgabe aufweist, ermuntern die Ergebnisse zu weiteren Anstrengungen in der Entwicklung abstraktiver TLS-Systeme. 

\newpage
\normalsize
\tableofcontents

\end{titlepage}

\pagenumbering{arabic} 


\chapter{Introduction}

Many newsworthy stories do not happen in a single day, but often develop in the course of multiple weeks, months or even years. In these long timespans many events occur that contribute to the larger story. Often information about these incidents may be spread out across a large number of different articles that add new information as the situation develops. This can make it difficult for users wanting to inform themselves about the events to get an overview of the larger story. Many online publications offer a solution to the problem by creating timelines of the events.

\begin{table}
\begin{tabularx}{\textwidth}{|X|}
\hline
\textbf{2011-03-15} \\\hline
First protests after calls on Facebook for a '' Day of Dignity . '' \\\hline
\textbf{2011-08-18} \\\hline
US President Barack Obama and his allies urge Assad to quit . \newline Western and Arab states later impose sanctions on his regime . \\\hline
\textbf{2011-10-02} \\\hline
Creation of the opposition Syrian National Council SNC . \\\hline
\textbf{2012-03-21} \\\hline
The UN Security Council demands that Syria implement a peace plan drawn up by UNArab League envoy Kofi Annan calling for a ceasefire and democratic transition . \\\hline
\textbf{2012-06-16} \\\hline
UN observers , deployed to monitor the April ceasefire deal , which was not respected , suspend operations . \\\hline
\end{tabularx}
\caption{Beginning of an example timeline about the Syrian civil war}
\label{tab:example-tl}
\end{table}

 %(EXAMPLES!)
A timeline lists dates which are important to the topic it covers along with brief summaries of what has happened at these dates. Table \ref{tab:example-tl} shows an example for such a timeline by the Australian broadcaster SBS.
While such timelines are very useful, they are usually created by hand and may not be available for all interesting topics. Even if there are timelines available, they might not cover the period of time interesting to the reader.
Consequently there has been a number of attempts to generate such timelines automatically from large corpora of news articles pertaining to one interesting topic. This task is called timeline summarization (TLS) and we review these approaches in section \ref{sec:prior-work-tl}.

As \citet{markert} note, TLS is closely related to Multi Document Summarization (MDS).
A Multi Document Summarizer is a system that takes documents about some common topic and outputs a summary of their content in a concise form.

Generally, there are two classes of MDS systems: abstractive systems and extractive systems \citep{a-brief-survey}. Extractive systems work by selecting a subset of sentences from the original documents that best represent their content to form a summary. This is the type of approach that has been employed by all TLS systems we are aware of so far.
There are, however, inherent limits to the performance of purely extractive systems. While a recent study \citep{hirao+nishino} using an oracle summarizer to determine an upper bound for extractive summarization has shown that there is still room for improvement on the usual summarization metric ROUGE \citep{rouge}, another study by \citet{ceylan+mihalcea} suggests that achieving meaningful improvements may be difficult at least for single document summarization. Their approach exhaustively generates all possible extractive summaries for documents in multiple domains and generates a probability density function from a histogram of ROUGE scores. They find that both the basic lead baseline and the TextRank system \citep{textrank} already fall into the 99th percentile of the function.

Abstractive summarization systems on the other hand are closer to the way we would expect a human to summarize a text. Instead of just selecting sentences from the original documents, they construct new sentences based on the information in the input documents. While they are not limited in the same way as extractive summarization systems, they are more difficult to design \citep{recent-advances, a-brief-survey}.
There are, however, a number of systems available that have been able to tackle the task with some success, although it is still very much an open problem \citep{recent-advances}. We give an overview over these systems in section \ref{sec:prior-work-mds}.
To our knowledge, no abstractive MDS system has been applied to TLS before.

%However, a number of abstractive systems has been proposed that tackle the task, for example by not constructing new sentences from scratch, but by merging sentences from the corpus and deleting unimportant constituents. To our knowledge, no such system has thus far been applied to TLS. Considering the availability of such systems for summarization and the limits of extractive summarization, it is interesting to investigate the application of such systems to the problem of timeline summarization.

A very basic approach to introduce abstractive timeline summarization is to take an existing abstractive MDS system and couple it with an algorithm that determines important dates within an event. This process is called date selection. We elaborate on this idea in section \ref{sec:baseline}. This approach, however, ignores the temporal aspect of TLS, which has previously shown to be important to the problem. For example, \citet{markert} have shown that temporalizing similarity functions used in an extractive summarization system can improve over just using the MDS system outright.

We thus aim to make the following contributions with this thesis:

\begin{enumerate}
\item{Evaluate the performance of using a well-performing abstractive MDS system to conduct abstractive TLS}
\item{Evaluate the potential for improvements over these scores by using the abstractive MDS system as a starting point and adapting it to TLS}
\item{Compare the performance of our system to an extractive approach}
\item{Evaluate the grammaticality of the system output}
\end{enumerate}

While the first two points will help to answer the question of whether abstractive TLS requires significant changes to our abstractive MDS baseline system, the second two points concern the general feasibility and promise of investigating abstractive TLS.

The rest of this work is structured as follows:
In section \ref{sec:prior-work} we give an overview over prior work in both TLS and abstractive MDS.
In section \ref{sec:baseline} we outline the abstractive MDS system we have choosen as a baseline and detail our improvements to this system for TLS in section \ref{sec:system}.
We evaluate both systems following the setup described in 
section \ref{sec:setup} and report the results of an automatic evaluation in section \ref{sec:results-auto}.
The setup and results of our manual grammaticality evaluation are presented in section \ref{sec:results-manual}.
We then give a brief insight into the output of our system and some common problems we have encountered with it, in section \ref{sec:errors}, before we come to a conclusion in section \ref{sec:conclusion}.

%We thus structure our investigation into abstractive TLS as follows:
%Starting from a state-of-the-art abstractive summarization system we first apply it to the TLS task %without modifications, in a pipeline with a date-selection algorithm.
%We then derive a general frameworks for abstractive TLS based on this original system and propose a %number of adaptions to take advantage of the special structure of the TLS task.
%Both the unmodified system and the adapted version are evaluated on two timeline summarization %corpora. We also conduct an annotation study to determine the linguistic quality, which abstractive summarization systems, unlike their extractive counterparts, can not guarantee even on sentence level.


\chapter{Prior Work} \label{sec:prior-work} 

\section{Prior Work in Abstractive Document Summarization} \label{sec:prior-work-mds} 

There is a great variety of models for abstractive summarization. One common approach is to shorten existing sentences by removing parts that are deemed not important. Usually this is combined with some kind of learned function that determines which deletions lead to good summaries.
The system of \citet{berg-kirkpatrick} is one example for this approach for multi-document summarization. They work on constituency trees that they compress by cutting edges to remove subtrees which results in compressed sentences.
They describe a valid summarization of an input document set $x$ as a binary vector $y$. Each entry in $y$ corresponds to one non-terminal node in one of the sentences of $x$. They set $y_i = 1$, if the $i$-th non-terminal should be included in the summary and $y_i = 0$ else. They impose constraints on $y$ so that removing a parent from the tree implies removing all its child nodes as well to ensure consistency.
To find a summary for $x$ we must then solve the following optimization problem over the set of all valid summary vectors $Y(x)$:
\begin{equation}
\max_{y \in Y(x)} \sum_{b \in B(y)} v_b + \sum_{c \in C(y)} v_c
\end{equation}
where $B(y)$ is the set of bigrams that are contained by the summary described by $y$ and $v_b$ is the reward for including the bigram $b$ in the summary and $C(y)$ is the set of cut choices the system has made, i.e. the set of edges that have been removed from an included sentence. $v_c$ is the reward for making cut choice $c$.

The first term guides the system towards covering the content of $x$ without being redundant. It rewards a summary for including bigram from the original document but punishes redundant sentences by counting each bigram only once.
The second term guides the abstractive part of the system, by rewarding the removal of unimportant parts of the sentence, i.e. those with high reward.

%The first term rewards the system for covering the content of $x$. $B(y)$ is the set of bigrams that are contained by the summary described by $y$ and $v_b$ is the reward for including the bigram $b$ in the summary. This corresponds to the bigram coverage model which is also used in extractive summarization \citet{gillick+favre}. Intuitively it rewards a summary for including bigram from the original document but punishes redundant sentences by counting each bigram only once.
%The second term guides the abstractive part of the system. $C(y)$ is the set of cut choices the system has made, i.e. the set of edges that have been removed from an included sentence. $v_c$ is the reward for making one such choice.

The rewards $v_b$ and $v_c$ are based on a feature function $f(x, y)$ and parameterized by a weight vector $w \in \mathbb{R}^n$.
The feature function $f(x, y)$ for a summary $y$ is defined in terms of two other feature functions: $g(b, x)$ encodes the features of a bigram $b$ and $h(c, x)$ encodes the features of a cut choice $c$. $f(x, y)$ is then the sum of the features for all bigrams and all cut choices in $y$:
\begin{equation}
f(x, y) = \sum_{b \in B(y)} g(b, x) + \sum_{c \in C(y)} h(c, x)
\end{equation}
For bigrams the features include the number of mentions in the corpus and the position of the first mention of the bigram in any of the documents.
For cuts the features describe the phrase types in the cut subtree.

To learn the parameter $w$ \citeauthor{berg-kirkpatrick} train a soft-margin SVM for structured learning \citet{svm} with bigram recall on a reference summary as the loss function. To ensure that the system is theoretically able to reproduce the target summaries, the training corpus is not composed of fully abstractive summaries, but instead of extractive summaries that were selected for maximum coverage of the input documents and shortened by human annotators.

While the inference phase of the problem can be stated as an ILP, \citet{berg-kirkpatrick} find that the space of possible cut choices makes an exact solution infeasible in many cases. They thus use an approximation procedure. Instead of selecting the sentences directly from the whole corpus, they first create a shorter extractive summary based on bigram coverage only and then run the system on this summary. By creating smaller intermediate summaries, they can reduce computation time for difficult cases at the expense of summary quality.

%\citet{almeida} try to solve this complexity problem in a similar setting by using 


%They cast the problem of summarizing the document set $x$ as searching the set $Y(x)$ of all indicator vectors describing a summary 
%They formulate the problem of summarizing a document set $x$ as finding an optimal indicator vector $y \in Y(x)$, where $Y(x)$ is the set of the descriptions for all possible summaries for $x$.

%For example, \citet{almeida} work on dependency trees and solve an ILP (WRONG) to determine both which sentences should be included in the summarization and which nodes can be deleted based on a learned scoring function. They use a multi-task learning setup to learn appropriate scores using both corpora for extractive summarization and sentence compression to increase available training data. Similarly \citet{berg-kirkpatrick} work on parse trees and also solve an ILP to jointly choose sentences to extract and nodes to discard from a sentence based on scores learned on a training set.

%Other approaches allow for more complex operations on the sentences.
While the system described before can only compress a single sentence, other approaches have tried to fuse multiple sentences to generate a single sentence that summarizes their content.
For example, \citet{bing} extract NPs and VPs from different sentences and recombine them to generate new sentences.
To find candidate phrases to combine, they filter valid combinations based on their \textit{compatibility}. To identify compatible phrases, they first group coreferent NPs together. Similarly, VPs are grouped together based on n-gram and named-entity overlap. Both sets of similar phrases form what \citeauthor{bing} call \textit{alternatives} to each other. A NP node $q$ and a VP node $p$ are compatible if either $q$ is an alternative to an NP that has the same parent node as $p$, or $p$ is an alternative to a VP node that has the same parent node as $q$. A single NP from the set of alternative is selected as a subject and can be combined with multiple VPs, which follow it. VPs are heuristically connected by inserting coordinations.

Their summarizer tries to optimize the salience of the included summaries. Salience is computed based on the number of n-gram concepts and entites included in the phrase, which is similar to how \citet{berg-kirkpatrick} score their summaries.
They use an ILP to find an optimal combination of NPs and VPs.

\citet{barzilay+mckeown} introduce a sentence fusion system, which allows for more generic recombination operations. They first cluster similar sentences into a number of \textit{themes} that consist of sentences covering similar information.
Themes are then ranked based on factors such as their size and the similarity and importance of their member sentences. The latter is computed based on the concept of lexical chains, which have previously been used for extractive text summarization \citep{lexical-chains}. Only the top $n$ themes are retained, the rest is discarded.
\citeauthor{barzilay+mckeown} then run a sentence fusion algorithm on each theme.
At the core of their fusion algorithm is the alignment of subtrees in dependency parses of the sentences.
By considering both the structure of the trees and the content of the nodes, they align subtrees from different sentences which cover the same content.

To generate new sentences, they build what they call a \textit{fusion lattice}. A fusion lattice is a dependency tree which is augmented with alternative realizations of the same informative content. This is similar to the concept of alternative phrases used by \citet{bing} described above.
A fusion lattice is built starting from a base tree, which guides the fusion proces. The base tree of a theme is selected by finding the sentence whose tree is on average best aligned with all other sentences in the theme.
The base tree is first augmented with subtrees from other sentences that it aligns well with. Nodes that describe the same entity, for example, may be merged together in one node. Finally, extraneous subtrees that were not part of a good alignment are cut.
The resulting lattice now has a variety of alternative realization of subtrees and nodes.
To generate a summary from the sentences, they score all possible alternative realizations of the lattice using a language model and output the alternative which achieves the best score.

\citet{filippova} has a similar approach to fusion, which she applies to generate abstractive headlines. She also fuses similar sentences to generate one sentences that subsumes their shared information. Unlike \citet{barzilay+mckeown} she works only on the sequence of words and their POS-tags and not parse trees. She models the set of input sentences as a directed graph of word adjacency relations, where sequences shared between sentences are mapped to the same nodes. Good summaries are expected to correspond to the shortest paths through this graph.
\citet{banerjee} use this method to create a MDS system. Besides the difference in sentence generation, their work has two more crucial differences to \citeauthor{barzilay+mckeown}:

\begin{itemize}
    \item{Clusters are not selected beforehand, but instead selected jointly with the sentences that summarize them.}
    \item{The generated sentences are given one score that combines their informative content and their linguistic quality. All possible candidates are considered for inclusion in the timeline at the same time based on this score, instead of one candidate being selected individually per cluster based on linguistic quality alone. Note, that each cluster may still contribute only one sentence to the summary.}
\end{itemize}

As the system of \citet{banerjee} forms the basis of this work, we describe it in more detail in section \ref{sec:baseline}.


%\citet{bing} also work on parse trees but combine NPs and VPs from different sentences if they are similar enough. They select the best combinations based on salience of the included phrases.

%Another approach to this kind of summarization, using the original documents to form new sentences, is based around graph modelling. \citet{filippova} suggests a model for %headline generation that is based around finding shortest paths in a word graph derived from a set of sentences.
%Each node is one of the words from the original sentences and directed edges indicate adjacency of two words in at least one of the sentences. Edges are weighted so that more %frequently cooccuring words are connected by shorter edges. By running a k-shortest path algorithm, \citeauthor{filippova} generates a set of candidate summaries, which are %then reweighted according to their length.
%\citet{banerjee}, which we base our investigation on, builds on this approach and integrates this method in an abstractive multi document summarization, by first clustering %the sentenences of the input documents and then using the graph compression method to generate a large number of candidate summary sentences for each cluster. They then %formulate an ILP to select the most salient sentences for inclusion in the summary. See section \ref{sec:mds-baseline} 
%
%The idea of building graphs from sentences has also been explored in other domains, for example \citet{opinosis} extract short summaries from highly opinated sentences from product reviews by using a similar method.


Finally, a very recent avenue of research is the use of neural networks for abstractive summarization. A typical architecture is to use an encoder-decoder architecture with some kind of recurrent neural network, for example an LSTM. See \citet{nallapati, rush} for examples of such approaches.
While many early models were limited to summarize only short passages of texts, recent developments combining this architecture with methods from reinforcement learning have substantially increased performance of neural summarizers for longer summaries. \citep{paulus}



\section{Prior Work in Timeline Summarization} \label{sec:prior-work-tl} 

%Similar Tasks
%Swan and Allen

TLS is related to a couple of different tasks, that all might reasonable be referred to as ''Timeline Summarization''.
The SemEval 2015 conference had a task called ''TimeLine: Cross-Document Event Ordering'' \citep{semeval-timeline}. Participating systems were given a named entity as a query and had to extract all events that entity was a part of from a corpus of news documents. Events also had to be anchored with a time tag.
While this is superficially similar to our task, there are some crucial differences:

\begin{itemize}
    \item{Systems only had to detect coreferent events and mark their textual extent. No human readable summaries were generated. }
    \item{Systems were asked to extract all events. Events were not ranked by importance and there was no limit for the number of events included. }
\end{itemize}

It will become clear in the rest of this thesis, that both aspects are an important part of the challenges of TLS.

Another related task is that of update summarization, which has been part of the TREC conference \citep{trec-update}.
In update summarization, systems are presented a stream of documents.
At regular intervals, the system may add sentences to a summarizations of the information received so far.
Crucially, however, once it has generated a sentence, it can not change it back.
While this task also requires the identification of important events in huge corpora of input documents, it is different from TLS, where timelines are optimized globally. For this reason, update summarization is also sometimes called \textit{online} summarization, while TLS is an \textit{offline} summarization task.

\citet{swan+allen-temporal-summaries} have formulated a setting similar to this. Instead of requiring a binary decision of whether a sentence should be included in a summary or not, they assign a score to each sentence, where more important sentences should receive a higher score. Similarly to which we would expect in an update summarization setting, they assign these scores directly when processing an article and not globally.

They formulate two measures for sentence importance: usefulness and novelty.
Usefulness measures to what extent the sentence covers the topic of our summarization. Novelty measures the extent to which a sentence introduces new information into the summary, that was not covered by other sentences before.
A good summary optimizes both.

They evaluate two variants of both measures.
Their best measure for usefulness is based on computing the probability of a sentence $s$ being generated by a language model $LM(D)$ infered from the article $D$ sentence $s$ came from:

\begin{equation}
P(s|LM(D)) = (\prod_{w \in s} \frac{\mathit{tf}(w, D) + 0.001}{1.01 * |D|})
\end{equation}
where $\mathit{tf(w, D)}$ is the frequency of word $w$ in document $D$.

Intuitively, this function rewards a sentence, if it has a high overlap with frequent words in the document. Assuming that each document we receive refers mostly to the topic of our summarization, this means $s$ contains useful information about that topic.

The best variant for novelty detection is also based on language modeling.
They assume that a sentence is novel, if it is unlikely to be generated from language models fitted on sentences that were already included in the summary. To allow for a reliable computation of the language models, sentences which have similar content are clustered and the language models are fitted on these clusters.

A more recent example for such a system is \citet{salient-updates}, an update summarization system for the domain of disaster summarization.
In their setting, they receive a batch of sentences each hour, which they then score for their salience. They add sentences to their summary, if they are both novel and salient.

They compute salience based on a number of features. Notably, they also compute a form of sentence \textit{burstiness} which is characterized by a sudden change in the tf-idf score of a sentence in the last 24 hours. This tries to capture situations when certain terms suddenly increase in importance in the stream. This is similar to the metric of the same name in TLS, which we describe further below.
% Compare Lang Model

We now turn to systems which employ the same general setup for TLS, as our system. In particular, they fulfill the following two criteria:

\begin{enumerate}
    \item{The generated timelines are a collection of individual summaries that each cover a specific point in time during a larger event. Summaries have a date-tag that identifies, which date of the event they summarize.}
    \item{The summarizers have full access to all documents during timeline generation.}
\end{enumerate}


One of the earliest approaches to timeline summarization, which fullfills these criteria and still serves as a baseline for extractive MDS in many cases (see the evaluation of \citet{markert}, for example), are the measures proposed by \citet{chieu}.
Their system takes a corpus $C$ of documents and a query $q$ which guides the summarization. Only sentences which match $q$ are considered. We write $C_q$ for the portion of the corpus that matches $q$.

\citeauthor{chieu} propose two measures to determine the importance of sentences for a timeline: \textit{interest} and \textit{burstiness}.
Both are based around the concept of two sentences reporting the same event. \textit{Interest} is based on the intuition that a sentence $s$ is important, if there are many other sentences in the document that refer to the same event as $s$. \textit{Burstiness} builds on the expectation that an important event leads to a short ''burst'' of sentences reporting the same event.
%In practical terms, interest is the measure of overall similarity of a sentence $s$ to those sentences in the corpus which lie in a specified window around the date of the event $s$ describes.
It is measured by determining whether a sentence reports an event that is strongly clustered around one specific date in the corpus, instead of spreading out evenly over its entire timespan. This is done by computing the log-likelihood ratio of the distribution of sentences reporting the same event as $s$ in a time window around $s$ and the distribtuion of these sentences outside of the window.
This distribution can be described by the contingency table \ref{tab:chieu}, where each cell contains the number of sentences that meet the criteria of their row and column.

\begin{table}
\begin{tabular}{|l|c|c|}
\hline
& same event as $s$ & $\lnot$ same event as $s$ \\
Within $k$ days of $\mathit{date}(s)$ & $a$ & $b$ \\
$\lnot$ Within $k$ days of $\mathit{date}(s)$ & $c$ & $d$ \\
\hline
\end{tabular}
\caption{Contingency Table for Burstiness}
\label{tab:chieu}
\end{table}

To match sentences to a date, each sentence needs to be assigned a date tag. This tag is determined either by the first date expression in the sentence or the time of document publication, if no such expression can be found in the document. \citeauthor{chieu} develop a small rule-based system for this by hand, because more complex (yet still rule-based) stand-alone solutions for date tagging such as Heideltime \citep{heideltime}, which is used by modern TLS systems \citep{tran-headlines, markert}, were not yet available at the time.


Because there is no easy way to decide automatically whether two sentences report the same event or not, \citeauthor{chieu} instead use a similarity score to determine sentences which report the same event. They compute the similarity of two sentences $s, s'$ using cosine distance between their vector representations $\vec{s}, \vec{s'}$: $\mathit{sim}(s, s') = \cos(\vec{s}, \vec{s'})$.
As cosine similarity is not reliable enough for their purpose, they filter sentences whose date tags are further apart than some threshold $T$ of dates.

They then define interest as:

\begin{equation}
\mathit{Interest}(s, T) = \sum_{s' \in C_q | |\mathit{date}(s') - \mathit{date}(s)| \leq T} \frac{\cos(\vec{s}, \vec{s'})}{\mathit{numdays}}
\end{equation}

Here, numdays is a normalization factor that is defined as $\text{number of days in dates} + 2 * T$ to account for the fact that $\mathit{date}$ can also refer to an entire week or month.

Similarly for burstiness they set:

\begin{align}
a = \sum_{s' \in C_q | |\mathit{date}(s) - \mathit{date}(s')| \leq T} \mathit{cos}(\vec{s}, \vec{s'}) \\
b = \sum_{s' \in C_q | |\mathit{date}(s) - \mathit{date}(s')| \leq T} 1 - \mathit{cos}(\vec{s}, \vec{s'})
\end{align}

$c, d$ are defined analogously for sentences outside the $T$-day window. The score for burstiness is obtained by computing the log-likelihood-ratio for $T \in 1, ..., 10$ and selecting the largest value.

To select sentences for the timeline, they order sentences according to one of the two metrics and greedily select sentences up to a fixed total number of sentences.
They avoid redundancy by computing what they call the \textit{extent} of each included sentence. When selecting a sentence $s$, all sentences that have not yet been selected and whose date is less than extent days away from $\mathit{date}(s)$ are removed from the pool of candidates. Extent is larger for sentences that have higher scores than other sentences in their neighbourhood.
This also means that the summary for each date included in the timeline is exactly one sentence long.

They evaluate their system on a corpus constructed from the Gigaword Corpus \citep{gigaword} and a set of manually constructed timelines. The corpus contains articles about several political leaders of mayor nations. Annotators were asked to rank gold and system timelines for the same query.
They find that their interest metric generally performs best.

When moving to longer timelines, that contain more than one sentence per daily summarization, it becomes important to model both the interaction of sentences within the same daily summary and the interaction between summaries of different dates.
\citet{yan-trans} model this relation by casting the summarization as a graph-centrality problem.
The importance rank of a sentence from the set of input documents is assessed by finding its centrality in two sentence graphs, a ''global-biased'' and a ''local-biased'' graph.
Intuitively the rank in the local graph indicates importance according to other sentences with the same date, while the global rank is the importance of a sentence relative to sentences with other dates. This follows the intuition that a good sentence in a timeline summarizes the events of the local date, but is also related to the summaries of other dates. For each date $d$ in the corpus, a local and a global graph is constructed.
Similary to \citet{chieu}, their notion of importance is based on cosine similarity to other sentences.

For all sentences with date tag $d$ the local graph is constructed by computing pairwise cosine similarity between sentences with the same timestamp and adding edges with appropriate weights between each pair of sentences.
The global graph is constructed from all sentences in the corpus, not only those with date tag $d$. Because the graph is supposed to reflect the global importance of each sentence from $d$, no edges are added between sentences with date tag $d$.
For all other sentences $s'$ the edge weight is defined by the cosine similarity of $s$ and $s'$ and the distance between their dates $|\mathit{date}(s) - \mathit{date}(s')|$. However, unlike \citet{chieu}, the influence of the sentence $s'$ on the importance of $s$ is not binary, but instead gradually decreases with temporal distance according to a kernal function $\Gamma(|\mathit{date}(s) - \mathit{date}(s')|)$. The further away $s'$ is from $s$, the lower its influence becomes.

The final rank of a sentence is determined by the weighted average centrality rank of a sentence in both graphs.
To create the final summary, the highest-ranking sentences are selected for a timestamp. To achieve compression, \citeauthor{yan-trans} specify a global compression rate $\varphi$. The ratio of available sentences that is allocated to the summary of date $d$ is $\frac{|C^{(d)}|}{|C|}$, where $C$ is the corpus of all sentences and $|C^{(d)}|$ is the number of sentences with date $d$.
This procedure is a simple way of identifying \textit{bursty} dates and allowing the summary to put more focus on them.
Note that this allows for more than one sentence per timestamp, unlike the work of \citet{chieu}.

The procedure of \citet{yan-trans} does not have an explicit facility to avoid redundancy and instead relies on the property of the algorithm they use to determine centrality, DivRank \citep{divrank}. DivRank is a variant of the well-known PageRank \citep{pagerank} algorithm. It modifies the original PageRank equation, so that a node with high centrality score becomes a sink for the probability mass allocated to neighbouring nodes. This means that nodes next to more important nodes lose probability mass, instead of profiting from their important neighbour. The procedure is designed to make sure that the most important nodes are not tightly clustered in one area of the graph, but instead spread accross the whole graph. Consequently, we can expect the DivRank algorithm to select sentences from different parts of the graph that are sufficiently diverse to avoid redundancy.

\citet{yan-trans} use a corpus of crawled articles from several internet newsmedia and timelines from the same sites that were created by human editors.
They use the ROUGE \citep{rouge} metric from traditional summarization to evaluate their timelines against the human-authored ones.

\citet{tran-headlines} formulate the task in a slightly different way again and only extract headlines from articles, instead of entire sentences. They argue that this procedure produces more understandable summaries.
They rate headlines based on their \textit{influence} and \textit{spread}.
The influence of a headline published at date $d$ is the extent to which sentences from other dates that refer to $d$ mention the event described by the headline.
This is modeled in terms of the similarity of word distributions. Let $\Theta(h)$ be the word distribution of the headline published at date $d$ and let $\Theta(\mathcal{E}_{V \rightarrow d})$ be the word distribution of all sentences that refer to date $d$, then the influence of the headline is

\begin{equation}
I(h) = \sum_{w \in h} p(w|\Theta(h)) * p(w|\Theta(\mathcal{E}_{V \rightarrow d}))
\end{equation}.

The \textit{spread} is a similarity-based metric. This rests on the assumption that headlines that are similar to many others must be important, because they are repeated so often, which is a notion that we have repeatedly encountered in the works described so far.
\citeauthor{tran-headlines} use a regression model trained on an external corpus to determine the \textit{duplication probability} of two headlines, which is a score between $0$ and $1$ that determines the probability that two headlines describe the same event.

Like \citet{yan-trans}, \citet{tran-headlines} model the scoring problem as a graph centrality problem where similar headlines are connected, although only one graph is built for headlines from all dates.
Edges between headlines are weighted using their duplication probabilities.
They compute scores using the personalized PageRank algorithm \citep{topic-sensitive-pagerank}, which updates the relevance score $R^{(t)}$ for headline $h_j$ at iteration $t$ according to the following formula:

\begin{equation}
R^{(t + 1)}(j) = d * \sum_{i} \frac{p_{ij}}{\sum_{k} p_{ik}} * R^{(t)}(i) + (1 - d) * \frac{I(h_j)}{\max_{h \in H} I(h)}
\end{equation}
where $d = 0.85$ is a dampening factor, $H$ is the set of all headlines and $p_{ij}$ is the duplication probability of headlines $h_i$ and $h_j$.

Sentences are ranked by their relevance $R$ and selected greedily under date count and daily summary length constraint.

% Formula?
% date tagging?



%\citet{yan-trans} model the summarization as a graph centrality problem. For each date, two separate graphs, are constructed: a local biased graph, that encodes the %similarities of sentences refering to that date and a global biased graph that encodes the similarities of the sentences of that date (''intra-date similarities'') to all %sentences outside that date (''inter-date similarities'').
%They then use the DivRank \citep{divrank} algorithm to determine centrality scores for each sentence within one date for each graph, which are used to tank the sentences.
%A final rank for each sentence is computed by a weighted average of the ranks obtained from both graphs. To determine which dates should be included in the timeline, they %give preference to sentences from dates which have been mentioned often in their corpus. As they themselves note, this corresponds to the notion of burstiness introduced by \citeauthor{chieu}.

All of the above systems have in common, that their scoring functions are entirely hand-crafted and not learned, though some systems have also adopted a strategy where scoring functions are learned on the corpora.
For example, \citet{tran-headlines} build two seperate linear regression models, one to learn importance scores for dates and one for sententences.
Their features for date scoring mostly rely on the number of times a date is mentioned in the corpus.
For sentence extraction they use a variety of features, that include the TF-IDF scores of the words in a sentence, its length and whether it includes a time expression, among others.

They train on a binary objective for dates, where the objective is $1$, if the date is in one of the reference timelines and $0$, if it is not.
For a sentence $s$ that is part of a document published at date $d$, they fit a model on the maximum similarity of $s$ to any sentence in one of the reference timelines at date $d$.
They evaluate their system on a corpus of mutiple topics, each of which contains a number of human-authored timelines and articles crawled from mayor online news publications. This corpus is usually called \textit{Timeline 17} in literature. As we use it in our experiments, we outline its properties in section \ref{sec:setup-corpora}. The regressors are trained by cross-evaluation, where one topic is used for evaluation and the rest as training data.

\citet{multimodal} also learn relevance scores using the same leave-one-topic-out procedure as \citet{tran-headlines}, but alter the setting to a multimodal task by including information from an image search machine in their importance judgement.
Their learning strategy is based around the concept of collaborative filtering.
They first construct a matrix $M \in \mathbb{R}^{m \times n}$, which has one row for each of the $m$ sentences in the training corpus $C$. The first column of the matrix contains the ROUGE-scores of the sentences according to the reference summaries. The following columns contain the features computed for the sentences.

There are two types of features:

\begin{description}
\item[Textual Features:]{These include the publication date of the document of $s_i$, unigram occurences and occurences of Subject-Verb-Object tuples.}
\item[Image Features:]{For each sentence an image is retrieved using an image search machine with the sentence as input. The top-ranked result is encoded as a fixed-size vector by taking the activations of the last layer of a CNN for the image. The CNN used in their implementation was created by \citet{Simonyan+Zisserman} and trained on the ImageNet \citep{ImageNet} object recognition challenge dataset.}
\end{description}

To learn a scoring function on $M$, \citet{multimodal} decompose $M$ into two matrices $P \in \mathbb{R}^{k \times m}, Q \in \mathbb{R}^{k \times n}$, so that $M \approx P^T * Q$. Here $k$ is a hyper~parameter, that determines how many latent dimensions each matrix has and thus the number of model parameters. Intuitively, each row in $P$ can be thought of as an embedding of a sentence in $C$ in a $k$-dimensional vector space and the $i$-th row of P can be used to predict the appropriate value of feature $i$ from such an embedding.
The find $P, Q$, \citeauthor{multimodal} solve the following optimization problem:
\begin{align}
\min_{P, Q} \sum_{(i,j) \in M} (M_{(i,j)} - \vec{p_i}^T\vec{q_j})^2 + \lambda_P ||\vec{p_i}|| ^ 2 + \lambda_Q ||\vec{q_j}|| ^ 2 \label{multimodal-objective}
\end{align}

This formulation rewards solutions that can reconstruct $M$ well, while punishing solutions with large parameter values to avoid overfitting. The trade-off between regularization and better fit is regulated by the coefficients $\lambda_P, \lambda_Q$.
The objective \ref{multimodal-objective} is optimized using stochastic gradient descent (SGD).

During prediction, a feature vector is computed for each new sentence $s_i$, with the ROUGE column empty. The system then computes an embedding vector $p'$ for $s_i$ using the objective mentioned above. By computing the dot product of $\vec{p'}$ and the first column of the feature embeddings $\vec{q_1}$, we obtain a predicted ROUGE-score for $s_i$.
The sentences are selected from highest predicted score to lowest.

They evaluate their system on the \textit{Timeline 17} dataset, also using the ROUGE metric.


The overview of different systems provided above shows that there are many different approaches to TLS and many slight differences in the setting of the task. Interestingly though, while many acknowledge the similarity of the TLS task to MDS \citep{yan-trans, tran-headlines} and some even evaluate against simple MDS baselines \citep{yan-trans}, \citet{markert} note that there are no attempts to build a strong TLS system based on the foundation of a state-of-the-art MDS system. This means it is difficult for TLS to profit directly from the development of stronger MDS systems.

They thus take advantage of a sub~mo~dular summarization framework devised by \citet{lin+bilmes} and adapt it to TLS.
They are able to show that the MDS system performs well on the TLS task and can further be improved by making modifications to the system that take into account the special properties of TLS summaries.
As we base parts of our system on their sub~mo~du~la~ri~ty framework for TLS, we do not discuss their approach in more detail and instead refer the reader to section \ref{sec:evaluation-auto-setup}.

Their findings in two common TLS evaluation corpora show that while extractive MDS performs well on TLS, improvements can be made by making an effort to adapt it to the challenges of TLS.
This further encourages our investigation into using abstractive MDS systems for TLS.

%As becomes apparent from the above overview, there is a lot of variation in the way timelines are evaluated.
%While many systems use ROUGE as their primary metric, the application of ROUGE to timeline summarization is not  

%\citet{markert} evaluate a MDS system based on submodular functions on TLS corpora and present several constraints and modified scoring functions that adapt it to TLS.
%They score  


%Timestamps? Compression?

%Tran
%Markert


%\citet{multimodal} have proposed a multimodal approach to timeline summarization by integrating image information. They cast the task as a collaborative filtering task. During training, they construct a matrix of textual features and image embeddings derived from a deep neural network for each sentence in the corpus, including a column of their rouge scores. They then factorize the matrix into two embedding matrices, one for sentences and one for the features, which together approximate the original matrix. During prediction the rouge column of a sentence is left empty and a new embedding is computed for the sentence. By then multiplying that embedding with the feature embedding for rouge, they can derive a score for each sentence. They then include the sentences with the highest scores and order them chronologically.




\chapter{Abstractive Timeline Summarization as Abstractive Multi-Document Summarization} \label{sec:baseline}

\section{Overview}

Timeline summarization is closely related to the field of multi-document summarization. As already mentioned in the introduction, any multi-document summarization system can directly be used for timeline summarization by first selecting a set of dates the timeline should include and then summarizing the set of documents that were published on that date.
While this procedure is bound to ignore the interaction of individual daily summaries, as each summary is created in isolation, it is a convenient baseline for the problem.

However, while there are many different approaches to abstractive multi-document summarization, not all are immediately suited to the nature of current timeline summarization tasks. Neural methods, while recently successful in abstractive document summarization \citep{paulus}, usually require a large amount of training data, which is difficult to obtain given the limited number of reference timelines in current timeline corpora (see \ref{sec:setup-corpora} for an overview).
We thus focus on methods that work in an unsupervised way, without the need for large training corpora.

We opt to extend the work of \citep{banerjee} for two reasons:

\begin{enumerate}
\item{It does not require any training or parameter tuning.}
\item{It can easily be decomposed into contained subproblems, which can independently be adapted to the challenge of timeline summarization.}
\end{enumerate}.

In the rest of this section we describe the algorithm devised by \citet{banerjee} in more detail and also highlight any implementation decisions we made during our reimplementation of their system.

\section{Multi-Document Baseline System} \label{sec:mds-baseline}

The reference system of \citet{banerjee} we have choosen for our task generates summaries in three steps. The clustering phase, the sentence generation phase and the selection phase. In the clustering phase similar sentences of the input documents are grouped into clusters. In the second step a graph-based compression algorithm is used on the clusters and novel candidate sentences are generated to summarize the cluster content. In the last step, sentences are scored for informativeness and linguistic quality and an optimal non-redundant subset of candidate sentences is choosen as a summary for the document.
While \citeauthor{banerjee} suggest multiple variations to their system, we only reimplemented their best performing configuration. Note, that we leave out the sentence ordering step of the original system, because of the tearse structure of most timelines, where most events are described in only one sentence.

\subsection{Clustering}  \label{sec:baseline-clustering}

Given a set of documents $\bar{D}$, the clustering phase is divided in two distinct steps. In the first step, \citeauthor{banerjee} select what they call a most important document $D_{imp}$. This document is expected to contain the most important content of the input document set and is used to guide the actual clustering process. The second step is the assignment of sentences to appropriate clusters.

\citeauthor{banerjee} conduct an evaluation on the DUC2004 \citep{duc2004} and DUC2005 \citep{duc2005} corpora and report the best clustering approach is based on what they call document set similarity.
The document set similarity of a document $D \in \bar{D}$, is the cosine similarity between the vector representation of $D$ and the representation for the concatenation of all documents in $\bar{D}$.
Following \citeauthor{banerjee} we represent each document as a tf-idf vector. We compute the document frequency based on all documents in $\bar{D}$.

In the clustering phase, we initialize one cluster for each sentence $s_j \in D_{imp}$. For all sentences $s_i \in D, D \in \bar{D} \setminus \{D_{imp}\}$ from other documents we iterate over each sentence of $D_{imp}$ and find the sentence $s_j$ most similar to $s_i$ as determined by the cosine similarity of their vector representations: $\argmax_{s_j} \cos(\vec{s_i}, \vec{s_j})$. The sentence is then added to the cluster induced by sentence $s_j$. To exclude sentences $s_i$ from the final clustering that are irrelevant to the summarization and not very similar to any sentence in $D_{imp}$, $s_i$ and $s_j$ must have a minimum similarity to be clustered.

In our implementation, we choose the same threshold as \citeauthor{banerjee} and reject all sentences which have a similarity of less than $0.5$ to all sentences in $D_{imp}$.

The last step of the clustering method is to filter out any unimportant clusters. This is achieved by eliminating all clusters which do not have at least $\frac{|D|}{2}$ members.

Note that in the rest of this work, we use the term document-based clustering to describe this procedure.


\subsection{Candidate Generation} \label{sec:baseline-generation}

The candidate generation step uses the graph based method for sentence generation of \citet{filippova}. The method is based on the idea that one can generate a good summary from a cluster of sentences $C$ by constructing a graph $H$ of words and their adjacency relations. Each sentence in $C$ corresponds to a loopless path from a special start node to a special end node in $H$. Sequences of words that are shared between two sentences are mapped to the same set of nodes and edges in $H$. By finding new paths through $H$ we can generate sentences combining elements from different sentences from $C$ to form a good summary of the cluster.

More in detail, \citeauthor{filippova} incrementally grows a directed graph $H$ of word adjanceny relations. Each word is represented as a tuple of surface form and POS tag. When a sentence $s$ is added to $H$, we test for each word $w \in s$ of the newly added sentence if it matches one of the vertices in the graph. If it does, it is mapped to that vertex. This is done in three phases, for three groups of words:

\begin{enumerate}
\item{Content words which have exactly one matching node in the graph}
\item{Content words which have more than one matching node in the graph, or which occur multiple times in the sentence}
\item{Stopwords and punctuation} % Punct not in filippova!
\end{enumerate}

If there is more than one vertex the word could be mapped to, we choose the vertex with the greatest overlap. Overlap is computed by checking if any of the predecessors and succesors of the candidate vertex in $H$ match the neighbouring words of $w$ in $s$.
Stopwords and punctuation are only mapped to a node, if they have an overlap of at least $1$.

If no match can be found for a node, a new node is introduced to the graph. Note, that each node can only be mapped to once.

After the mapping and optional creation of new nodes, edges are introduced between every two nodes that two neighbouring words were mapped to.
Finally, we introduce edges from a special start node to the node of the first word in the sentence and from the node of the last word to a special terminal node.

In order to find summarizations from $H$, we must find loopless paths from the start node to the terminal node. \citet{filippova} uses a k-shortest path algorithm to select promising candidate sentences. To steer the generation towards useful sentences, edges are weighted so it is cheaper to follow edges that were more frequently used in the construction step. Note, that while \citeauthor{banerjee} follow the same approach, it is unclear, if and how they weight edges.
In our reimplementation we choose the following function proposed by \citeauthor{filippova} to weight edges based on their frequency $\mathit{freq}$:

\begin{equation}
w(e_{ij}) = \frac{\mathit{freq}(i) + \mathit{freq}(j)} {\mathit{freq}(i) * \mathit{freq}(j) * (\sum_{s \in C} \mathit{diff}(s, i, j)^{-1})}
\end{equation}

where $e_{ij}$ is the (directed) edge that connects vertices $i$ and $j$ in $H$.

$\mathit{diff}(s, i, j)$ is a function that rewards edges between nodes that are often found close to each other in sentences. It is defined in terms of the position $\mathit{pos}(s, i)$ of the word in sentence $s$ that was mapped to vertex $v_i$.

\begin{equation}
\mathit{diff}(s, i, j) = \begin{cases}
\mathit{pos}(s, j) - \mathit{pos}(s, i) & \mathit{if }\mathit{pos}(s, i) < \mathit{pos}(s, j) \\
0 & \mathit{else}
\end{cases}
\end{equation}

After the k-shortest paths have been obtained, two heuristics are applied to remove ungrammatical sentences:

\begin{itemize}
    \item {Sentences with fewer than $8$ words are removed}
    \item {Sentences which do not have a verb are removed}
\end{itemize}

Following \citet{banerjee} we retain only the 200 shortest paths of each graph for computational reasons.

In our reimplementation we use the algorithm of \citet{kshortestpath} as implemented in the networkx library \citep{networkx} to generate the $k$-shortest paths.

\subsection{Scoring} \label{sec:baseline-scoring}

Given the set of generated sentences, we need to select the best candidates to be included in the final summary. \citet{banerjee} use two features to measure the quality of a sentence, linguistic quality and informativeness. This setup is supposed to balance readability of the sentences with their informative content.

Linguistic quality is measured by a trigram language model. Given such a model $P$ and a sentence $s = w_1, ..., w_n$ the lingistic quality $LQ(s)$ is defined as follows.
\begin{align}
LQ(w_1, ..., w_n) = \frac{1}{1 - (\frac{1}{L} \log \prod_{i > 3}^{n} P(w_i | w_{i - 1}, w_{i - 2}))}
\end{align}

The informativeness $I$ is measured by computing TextRank \citep{textrank} keyword scores and summing them over all tokens of a sentence.
TextRank scores are computed by applying the PageRank \citep{pagerank} algorithm to an unweighted cooccurence graph of the content words of the input sentences.

Given the TextRank score $TR^{(t)}(w)$ of word $w_i$ at iteration $t$, the score at iteration $t + 1$ is defined as

\begin{equation}
TR^{(t + 1)}(w_i) = (1 - d) + d * \sum_{w_j \in \mathit{adj}(w_i)} \frac{1}{|\mathit{adj}(w_j)|} * TR^{(t)}(w_j)
\end{equation}

where $\mathit{adj}$ is the set of words that have occurred in a window around $w_i$ in at least one sentence.
This corresponds to running the PageRank algorithm on an unweighted and undirected graph.

In our implementation we set the window size to cover the entire sentence.

Note that the TextRank scores are computed individually for each cluster, instead of globally for the entire document collection.

Given the final TextRank scores $TR(w)$, informativeness of a sentence $s$ is defined as

\begin{equation}
    I(s) = \sum_{w \in s} TR(s)
\end{equation}

The final score for a sentence $s$ is $\mathit{score}(s) = I(s) * LQ(s) * \frac{1}{\mathit{length}(s)}$, where $\mathit{length}(s)$ is the length in tokens of sentence $s$. %Note that we have slightly altered the original function by leaving out the division by length in the original formulation. This takes into account the different length constraints of our TLS setup and the DUC setup, which is used by \citeauthor{banerjee}. While DUC has a length limit in tokens, we instead limit the number of sentences for each entry.
%This means it becomes less attractive to formulate many small sentences, instead rewarding the formulation of longer sentences.


\subsection{Selection} \label{sec:baseline-selection}

Given the set of generated sentences $G$ and their scores, we now need to select a subset for inclusion in the summary.
Our baseline system uses an Integer Linear Program (ILP) to find a summary that maximizes the total score, which is a common technique in summarization tasks \citep{recent-advances}. %CHECK!

The objective is to optimize the following function:

\begin{equation}
\sum_{i = 1}^{i = |G|} \mathit{score}(s_i) * p_i
\end{equation}

where $p_i \in \{0, 1\}$ are binary indicators. $p_i = 1$ indicates sentence $s_i \in G$ should be included in the final summary.

To avoid redundancy, the following set of constraints is imposed:

\begin{equation}
\forall i \neq j \land \mathit{sim}(s_i, s_j) > t : p_i + p_{j} \leq 1
\end{equation}

Following \citet{banerjee}, we set $t = 0.8$ and use cosine similarities between vector representations of the two sentences for similarity computation.

A second set of constraints ensures that only one sentence from a cluster may be included in the final summary:

\begin{equation}
\forall s_i \in G: \sum_{s_j \in G | \mathit{cluster}(s_i) = \mathit{cluster}(s_j)} p_j \leq 1
\end{equation}
where $\mathit{cluster}(s)$ is the cluster sentence $s$ was generated from.

Finally, we also need to add an appropriate length constraint.
We introduce two alternative length constraints, a constraint of token count per summary and a constraint of sentence count per summary.
While the latter is common in TLS \citep{markert, yan-trans}, token count is usually used in MDS, for example for the DUC2005 conference \citep{duc2005}.

For the sentence length constraint we have
\begin{equation}
\sum_{i} p_i \leq k_{sent}
\end{equation}
where $k_{sent}$ is the maximum number of sentences per date.

For the token length constraint we have
\begin{equation}
\sum_{i} \mathit{length}(s_i) * p_i \leq k_{tok}
\end{equation}
where $k_{tok}$ is the maximum number of tokens per date.


Additionally we introduce another constraint to only include the 20 largest clusters at all steps to keep solving the ILP problem feasible.

\section{Adaption to Timeline Summarization}

As mentioned in the introduction, we can use the system described above as a TLS system by running a date-selection algorithm and then running the MDS system on the documents published at each selected dates.

While there are sophisticated solutions for date selection (see for example \citet{tran-datesel}, who model the problem as a random walk on a graph of dates), investigating date selection is not the primary focus of this work.
We thus reach for a simpler importance metric, which relies on the intuition that a date is important if it is mentioned often in the corpus. We select the $k$ most frequently mentioned dates for inclusion in the timeline.

%Date selection has been investigated both as part of end-to-end TLS systems (see, for example \citet{trans}), as well as a standalone problem.
%In this work we opt for a simple variant of the above mentioned principle and rank each date by number of mentions of the date in the corpus.

\chapter{An Abstractive Timeline Summarization System} \label{sec:system}

\section{Architectural Overview}

\begin{figure}
\includegraphics[width=\textwidth]{graphic}
\caption{A graphical oveview of our system architecture. Clusters are formed from the sentences in the corpus. Each cluster is assigned a date on which the event it describes supposedly took place. Parts of different sentences in the word-graph are combined to form novel sentences. Sentences are assigned a score and then selected for inclusion in the timeline}
\label{fig:arch-overview}
\end{figure}

Our approach to abstractive timeline summarization can broadly be divided into four distinct subproblems, generalizing the setup of \citet{banerjee}.
We take as input a collection of sentences $U$ from a set of documents $\bar{D}$.
We begin with the \textbf{clustering step} where we seek to group the sentences $U$ into a set of clusters $\bar{C}$. Each cluster should ideally represent one distinct event.

The second step is to generate a set of \textbf{candidate sentences} $G$. Each candidate sentence tries to concisely summarize the content of the source cluster it was generated from. Note that many more candidates are generated than can be included in the final summary. Each cluster candidate needs to be assigned a date, in order to match it to one daily summary. We do this by determining a date tag for its source cluster. The tag of a candidate sentence is the tag of the cluster it was generated from.

In the third step each generated sentence $s \in G$ is assigned a \textbf{score}, which represents its importance for the summary.
Finally, we must \textbf{select} a number of sentences from $G$ to be included in the timeline, so that the resulting summary is both informative and readable. See figure \ref{fig:arch-overview}, for a gaphical representation of these steps.

The baseline system also fits into this general schema, where the number of selectable dates is constrained beforehand by the date selection algorithm.

In the rest of this section we describe the methods we have developed within this framework for TLS. We present new approaches to clustering, scoring and selection. We do not, however, change the sentence generation phase as described in section \ref{sec:baseline-generation}.

\section{Clustering}

While there is a large body of work on clustering algorithms (see \citet{xu} for a relatively recent overview), the requirements of this application place some constraints on which algorithms we can consider. Sentences in one cluster must be both similar, so the compression can generate meaningful and correct sentences, as well as sufficiently different to allow for variations in the generated sentences, so that the system will not just reproduce the original sentences.
Furthermore the system must automatically detect the number of clusters on a wide varity of datasets with different sizes and characteristics.
Choosing too few large clusters may make it impossible to generate a sufficient amount of sentences to fill a timeline, because only one sentence is selected from the set of candidates generated for one cluster. Too many small clusters on the other hand may not be diverse enough to allow for novel sentences.
This requirement excludes, for example, the widely used k-means algorithm \citet{kmeans}, as it requires the number of clusters to be known beforehand.
These factors make the clustering step of the system quite challenging and at the same time an essential factor in improving the performance of the system.

We have divised two clustering methods, \textit{agglomerative multi-clustering} and \textit{affinity propagation}, which we describe in detail in the following two sections.

%\subsection{Document-based clustering}
%
%We can directly apply the clustering method of our reference MDS system as described in section \ref{sec:baseline-system} by treating each date %as an isolated clustering problem.
%Each date that has at least two articles in the corpus is treated as an independet clustering problem.
%We first determine the most-important document by finding the document which has the highest cosine similarity to the concatenation of all %documents in the date.
%Afterwards the sentences of the remaining documents are clustered to the most similar sentence of the most-important document, provided they have %a cosine similarity of $0.5$. Cosine similarity is computed based on a Tf-idf model fitted on the entire corpus of input documents.
%This setup corresponds to the best-performing system of \citet{banerjee}.
%
%One drawback of this clustering method is that it ignores possible date tagging on the sentence level and assumes instead that all sentences %refer to the publication date. 

\subsection{Agglomerative Multi-Clustering}

The document-based clustering approach detailed in section \ref{sec:baseline-clustering} assumes that all sentences in a document describe events on the same date, the document creation time. This assumption, however, ignores the more detailed date-tags that systems such as HeidelTime \citep{heideltime} can provide. While this is fine for MDS which mostly works with one event at one point in time, it is problematic for TLS, where we work with many events at many different points in time.

When moving to this more fine-grained date tagging level, we can no longer rely on documents to guide our clustering. Sentences that refer to one date may hail from many different documents and the sentences of one document can refer to many different dates. We demonstrate this property of our corpora in section \ref{sec:setup-stats}.
Consequently, we can also no longer rely on one document to guide the number of clusters as in the baseline system. We can, however, partly avoid the resulting problem of having to choose the number of clusters by relaxing the problem so that one sentence may be included in more than one cluster for the same date.
This means that we do not have to decide into which cluster a given sentence fits best, but just whether its inclusion would benefit a given cluster or not.

The reason for this relaxation becomes clearer, if we analyze the effects of placing a sentence in a cluster it does not belong to versus not placing an important sentence in a cluster that would benefit from it.
If we add a sentence to a cluster it does not belong to, it is, in the worst case, going to introduce a number of new paths to the graph, which do not correspond to good summaries. However, no paths that were previously possible are removed from the graph. This means we can filter these sentences out with good enough scoring methods.
The second problem, on the other hand, is much more difficult to correct at a later step. If a sentence carrying useful information is not included in a cluster, that information is lost to the summarizer. A sentence that corresponds to a path that does not exist in the graph can never be generated. It seems thus reasonable to choose a generation strategy that prefers making mistakes of the former kind instead of the latter.

Based on this observation, we formulate the following clustering strategy. We first partition all sentences $U$ of the input documents into sets $U_{d_1}, \dots U_{d_n}$, $d_1, \dots, d_n \in \{ date(s) | s \in U \}$ so that all sentences that refer to the same date are in one set. We assume that the correct date tag for each sentence is the first complete date tag encountered in the sentence, or the document creation time, if there is no such tag. While this is in line with prior work on TLS \citep{chieu, markert}, it is possible to improve on this assumptions, as we show in the next section.

We then consider each sentence $s$ in a block $U_d$ and build a cluster based on $s$ by finding similar sentences in $U_d$.
To construct the cluster for sentence $s$, we find every sentence $s' \in U_d$ so that $\mathit{sim}(s, s') > 0.5$ and add $s'$ to the cluster induced by $s$. Note that we have taken the clustering threshold directly from \citet{banerjee}. It was not tuned on the corpus.

We eliminate all clusters with only one member. We now have at most $|U_d|$ potential clusters for block $U_d$.
Many of the remaining clusters are bound to be very similar. While this is not a fundamental problem for the system, besides complicating redundancy avoidance in later steps, it is undesirable from a performance perspective, because it leads to a lot of unnecessary duplication of candidate sentences in the generation step.
We thus eliminate any clusters, which is a subset of another cluster in the same partition. See algorithm \ref{alg:multi-clustering} for a sketch of the whole process.


 \begin{algorithm}
 \caption{Agglomerative Multi-Clustering}
 \label{alg:multi-clustering}
 \begin{algorithmic}
    \STATE Partition the input sentences into sets $U_d = \{ s \in U | date(s) = d \}$ for each date $d$ in the corpus
    \STATE $\bar{C} = \emptyset$
    \FOR{all sets $U_d$}
        \FOR{$s_1 \in U_d$}
            \STATE Let $C = \{  s_1 \}$
            \FOR{$s_2 \in U_d \setminus \{ s_1 \}$}
                \IF{$sim(s_1, s_2) > 0.5$}
                    \STATE $C \leftarrow C \cup \{ s_2 \}$
                \ENDIF
            \ENDFOR

            \IF{$|C| \geq 2$}
                \STATE $\bar{C} \leftarrow \bar{C} \cup \{ C \}$
            \ENDIF
        \ENDFOR

    \ENDFOR
    \STATE $\bar{C} \leftarrow \mathit{EliminateDuplicates}(\bar{C})$
    \RETURN $\bar{C}$

\end{algorithmic}
\end{algorithm}


\subsection{Graph-based Clustering}

Both the agglomerative and the document-based clustering methods have in common that they assume that each sentence has an unambiguous date reference (either a date expression or the document creation date).
This assumption, however, is not always correct. Many date references are incomplete in that they do not refer to an exact date, but instead to a timespan. Take the following example from the Crisis corpus:

\countquote{
''A hospital source and an activist reported 17 dead in Deraa, where protests began last month before spreading across the country.''
\label{quot:hospital}
}{Source: Crisis corpus, document syria/2011-04-08/100.htm.txt}
 % corpora/syria/articles/2011-04-08/100.htm.txt: 

Instead of refering to an exact date, the article contains an approximate reference to ''last month''. While this construction would be recognized by a time expression tagger, such as HeidelTime \citep{heideltime}, many traditional TLS systems would discard this inexact reference, because it does not refer to an exact date.
Where it is handled, for example in the system of \citet{chieu}, it may be taken into account to determine the importance of other sentences, but it is usually not assigned an exact date.
A small corpus study on the \textit{Timeline 17} and \textit{Crisis} corpora we use for evaluation (see section \ref{sec:setup-corpora}, for a description of the corpora) presented in table \ref{tab:ambigous-date-ref} shows, that these constructions are not uncommon.

\begin{table}
\begin{center}
\begin{tabular}{|l|c|c|c|}
\hline
\textbf{Corpus} & \textbf{num. articles} & \textbf{num. inexact refs.} & \textbf{num. exact refs.} \\\hline
Crisis & 20457 & 61001 & 74702 \\
Timeline 17 & 4374 & 17655 & 23138 \\\hline
\end{tabular}
\caption{Frequency of exact and inexact date references in our evaluation corpora}
\label{tab:ambigous-date-ref}
\end{center}
\end{table}


Handling them becomes even more attractive when we consider that \citet{datesel-graph} have argued that events which are referenced a long time after they have happened are likely to be very important, because they have apparently remained in collective memory. %(CITE!)
%In the same vein, it might be useful to try to resolve date assignment for sentences with multiple date references instead of just discarding all date references besides the first.

Extracting the time an event has happened, has also been studied as a standalone problem. \citet{event-time-extraction} have developed a tree of neural classifiers, that tries to determine event dates based on the context of the sentence it is mentioned in and surrounding time expressions. Unlike our system, which is solely concerned with determining the exact date for inexact expressions, their system also allows to date sentences with no date tag at all, instead of just falling back to the DCT.
While it would certainly be interesting to test the applicability of such a system to TLS, we leave this investigation to future work, as it could easily be integrated later as a preprocessing step to our system.
We instead focus on an approach that integrates the event date determination for sentences without exact dates with the clustering step.

To determine the date an inexact reference in sentence $s$ refers to, we find similar sentences which have an exact date within the timespan referred to. If there are many sentences at some date $d$ that are similar to $s$, $s$ is likely to describe the same event as these and thus to refer to the same date. In order to do this, we cast the problem of sentence clustering as a graph-clustering problem where nodes are candidate sentences and edges are preferences for two sentences to be clustered together. By restricting which edges are included in the graph based on the dates referenced in the sentences they connect, we can ensure that sentences only end up in the same cluster if their dates match.

We define the function $\mathit{daterefs}(s)$ to return the set of date references of sentence $s$. These references include all TimeML-expressions \citep{timeml} contained in the sentence of type ''date'' and the document creation time of the parent document of $s$.
We call a reference $r \in \mathit{daterefs}(s)$ \textit{exact}, if it corresponds to the resolution of the timeline, i.e. the smallest unit of time the timeline covers in one summary. Usually this resolution is one day. We assume the document creation times are always exact references. All other references are \textit{inexact}.

%Intuitively this encodes that a sentence that mentions one ore more exact dates will refer to one of these dates. A sentence with one or more inexact tags might refer to a point in time inside the mentioned timespan, or it might refer to events happening at the time of publication. A sentence without any tags will always assumed to refer to the date of publication.
Intuitively this procedure encodes the assumption that a sentence that refers to an event in the past will contain some sort of reference to the date of that event, otherwise it is likely to refer to an event that has happened on the date of publication.
Note that we do not assume the converse. Just because a sentence includes some sort of reference to the past, its focus does not necessarily have to be about an event in the past. Instead it might just contain these references as background info.

To illustrate this intuition, we look at the example \ref{quot:hospital} again. We find that even though there is a reference to ''last month'' this is clearly not the date the 17 dead were reported.

Naturally it would be possible to extend the number of possible date references and including the tags of surrounding sentences as well, to account for situations were a past event is described over multiple sentences and the date is not repeated. We do, however, not include such references to restrict the number of edges in the resulting graph to a computationally feasible number.
%Note that this definition is not set in stone. For example one may assume that an untagged sentence would be likely to refer to any date mentioned in the same document. However, constructing a graph for two many date references quickly becomes computationally expensive, so we restrict ourselves to the above mentioned defintion.

To deal with the problem of inexact references, we also need the notion of one date reference being contained by another, for example, the second calendar week of 2013 is contained in the month January of that year.
A date reference $r_1$ is contained in a date reference $r_2$ if $r_1$ falls within the timespan covered by $r_2$, or both references reference the same timespan.

Using these notions, given a set of sentences $U$ we define a weighted, directed graph $H = (V, E)$ with weight function $w$ as follows:
\begin{align}
V = U \\
E = \{ (s_1, s_2) \in V \times V \mid \exists r_1 \in \mathit{daterefs}(s_1), r_2 \in \mathit{daterefs}(s_2) \colon \\ r_2 \text{ contains } r_1 \land r_1 \text{ is exact} \} \\ % Inexact to inexact edges?
w(s_1, s_2) = \mathit{sim}(s_1, s_2)
\end{align}

Note that while we introduce edges from inexact to exact references, we do not allow edges in the opposite direction. This means that while sentences with inexact references have a preference to be grouped with similar sentences with exact date references, this is not true the other way around.
This avoids that two sentences with different exact date references are grouped in the same cluster via the sentence with the inexact reference.

Having constructed the graph, we now need to find an appropriate clustering of the nodes in $H$. While we could theoretically use any algorithm that can cluster nodes in a directed graph, in this work we adopt Affinity Propagation for this task.

Affinity Propagation (AP) \citep{ap} is a clustering method that is based on the concept of message passing. It identifies a set of exemplars from the input instances which are the centers of the clusters. Each instance chooses its examplar based on its affinities to other points $s(i, k)$. A high affinity to a point $k$ indicates a high preference of $i$ to be clustered with $k$.
The total number of examplars is not fixed but is instead controlled by the diagonal of the similarity matrix $s(k, k)$, the \textit{preferences}. Larger values on the diagonal lead to the corresponding instance to be more likely to be an exemplar.
As we have no preference for the number of clusters, we set the diagonal to the median of similarities for each instance, as suggested by \citeauthor{ap}.
In each step of the algorithm, two matrices are updated: the responsibility $r$ and the availability $a$. The responsibility $r(i, k)$ indicates the degree to which the instance $i$ believes it should choose $k$ as an examplar over other potential examplars. It is updated according to the following rule:
\begin{equation}
r(i, k) \leftarrow s(i, k) - max_{k' \neq k}( a(i, k') + s(i, k') )
\end{equation}

The availability $a(i, k)$ indicates how appropriate it is for point $i$ to choose $k$ as an exemplar.
It is updated as follows for $i \neq k$:
\begin{equation}
a(i, k) \leftarrow \min 0, r(k, k) + \sum_{i' \not\in \{i, k\}} \max(0, r(i', k))
\end{equation}

For $k = i$ we have
\begin{equation}
a(k, k) \leftarrow \sum_{i' \neq k} \max(0, r(i', k))
\end{equation}

The procedure is then repeated until convergence.

Affinity Propagation is well suited to our task, as it does not need the number of clusters to be specified beforehand. Furthermore it has been used for clustering in summarization before \citep{salient-updates-disaster} and it can be implemented in a sparse manner so that each iteration of the AP algorithm is only linear in the number of similarities, instead of quadratic in the number of instances. Given that we expect most sentences not to be connected, because there is no overlap in their date references, this is a very useful property.

We further introduce a requirement for the similarity of two sentences to be at least $0.2$ \footnote{We have determined this threshold empirically based on performance considerations. We raised it until computations would complete in an acceptable timeframe on our machine. It was not tuned on the evaluation measures.} for two nodes to be connected. While this is not strictly required from a theoretical standpoint, it greatly speeds up computation of the clustering and reduces storage requirements for the graph.

As before, we compute the similarities using cosine similarity of the tf-idf vector representations of the sentences.

\subsubsection{Cluster Date Assignment}

To assign the candidate sentences to a summary, they need a date tag. Unlike the agglomerative clustering method, the date for a cluster created by this method is not clear. For example, we have no direct way of knowing whether a sentence was clustered based on the document creation time, or one of its date references.

We thus need to determine the appropriate date tag for a cluster $C$ after the clustering step. We choose a simple heuristic and count the number of times a date is mentioned in the corpus.

For cluster $C$ we assign each date $d$ a frequency $\mathit{freq}_C(d)$ with

\begin{equation}
\mathit{freq}_C(d) = \sum_{s \in C} \mathbbm{1}_{d \in \mathit{daterefs}(s)}
\end{equation}.

The date tag for the sentence is the exact tag with the highest frequency.

%\section{Candidate Generation}

%During candidate generation we largely rely on the same method and procedure that was used by \citet{banerjee} and outlined in section \ref{sec:baseline:details}. We do, however, change 

\section{Scoring} \label{sec:scoring}

Another avenue to improve the MDS system for timeline summarization is the scoring function.
In the following sections we propose a number of scoring functions that seek to adapt the existing functions to TLS and add new ones.
Following \citet{banerjee}, the final score of a sentence is the product of the values of the individual scoring functions.
While this is likely to run into numerical problems with floating point math for a larger number of features, we have not encountered this problem in our experiments.
We have conducted preliminary experiments with both a (unweighted) sum of the individual scores and a parameterized scoring function based on regression with the ROUGE score of each sentence as a target. Both, however, have yielded much worse results than the product.

\subsection{MDS Scores} \label{sec:mds-scores}

The scoring functions used in the baseline MDS system (see section \ref{sec:baseline-scoring}) can naturally also be applied to the timeline summarization task.
We refer to the linguistic quality score as $f_{ling}$ and to the informativeness score as $f_{TR^{(cluster)}}$.
We refer to the normalization by sentence length in the original scoring function as $f_{length}(s) = 1 / |s|$.
%It should be noted, however, that we do not use the  apply $f_{TR_{cluster}}$ score in isolation  

\subsection{Global TextRank}

The MDS functions described in \ref{sec:baseline-scoring} score candidates based on their importance in relation to the sentences in their cluster.
However, this score is not informative about the importance of the sentence in relation to the summarization as a whole.
It is thus reliant on some kind of filtering of irrelevant clusters. \citet{banerjee} solve this problem by removing clusters that are small in relation to the number of documents in the corpus. This solution, however, is difficult to transfer directly to timeline summarization, because there is no natural reference point for when a cluster is ''large enough'' to warrant inclusion in the timeline.

We thus seek to score the candidate sentences in relation to their importance for the whole summary, instead of the cluster.

A natural way to achieve this is to compute the TextRank \citep{textrank} scores on the concatenation of all documents in the corpus, instead of per cluster.
We refer to the score derived this way as $f_{TR^{(global)}}$.

\subsection{Temporalized TextRank}

While the global score gives an indication of the importance of a sentence in relation to the corpus, it does not respect that the importance of keywords may vary over time as the event develops.
We thus extend the TextRank scoring to be computed per date. We first collect all sentences in the corpus that pertain to one date, as per their cluster membership, and then run the TextRank algorithm to obtain keyword scores for all non-stopwords. We call this per-date TextRank $TR^{(date)}$.
This is different from $TR^{(cluster)}$ in that the sentences from all clusters that have the same date tag are considered, not just those in one cluster.

Previous research \citep{yan-trans, markert} has shown that scoring methods based on sentence similarity can be improved, if similarity is not only computed for sentences with the same timestamp, but for sentences with neighboring timestamps as well. The influence of sentences that are further apart is discounted based on the distance of their timestamps.
We build on this idea to temporalize the TextRank scores by allowing the scores of a date to be influenced by neighboring dates. The influence of neighboring scores is determined by the distance in days between the two dates and a soft cutoff function $g: \mathbb{N} \rightarrow \mathbb{R}$ that decreases with large date differences.
The final TextRank score is the average score of a word across all dates, weighted by this soft cutoff.

Let $V$ be the vocabulary of the candidate sentences and $E$ the set of all dates referred to in the corpus. We define the temporalized TextRank score $TR^{(temp)}_d(w), w \in V$ at the date $d$ follows:

\begin{equation}
    TR^{(temp)}_d(w) = \frac{\sum_{d' \in E} TR^{(date)}_{d'}(w) * g(|d - d'|)}{\sum_{d' \in E} g(|d - d'|)}
\end{equation}

We do not experiment with different choices for $g$ and set $g(x) = \frac{1}{\sqrt{x + 1}}$.

Using these values, we can define the scoring function $f_{TR^{(date)}}(s) = \sum_{w \in s} TR^{(temp)}_{\mathit{date}(s)}(w)$ for $TR^{(temp)}$, where $\mathit{date}(s)$ refers to the date that was assigned to the candidate $s$ via the cluster it was generated from.
The definition is analogous for $f_{TR^{(date)}}$.

\subsection{TLS Inspired Scores} \label{sec:tls-scores}

In addition to the improved MDS scores it might also be worthwhile to include scoring functions that have been shown to be useful especially in the TLS task.

One such metric is the number of times a date is referenced in the corpus \citep{markert, datesel-graph}. It is motivated by the intuition that a date that is mentioned often in the corpus is probably important for the event in question. We define the date reference score $f_{\mathit{dateref}}$ as the number of times the date of the sentence was mentioned divided by the number of sentences in the corpus $|U|$ for normalization:

\begin{equation}
f_{\mathit{dateref}}(s) = \frac{\mathit{refcount}(\mathit{date}(s))}{|U|}
\end{equation}

Another metric that has been used in both TLS \citep{chieu} and MDS (for example \citet{lexrank}) is the cosine similarity to the other sentences in the corpus.
Applying this metric directly to our problem is computationally expensive, because it requires $|U| * |G|$ similarity computations and $|G|$ may include more than a million sentences (see \ref{sec:setup-stats}).

A cheaper alternative is to use the size of the cluster a sentence stems from as a proxy. Clusters are assigned based on similarity and large clusters thus indicate that a sentence is very similar to many other sentences.

We define the $f_{\mathit{csize}}$ function as follows:

\begin{equation}
f_{\mathit{csize}}(s) = \frac{|\mathit{cluster}(s)|}{\max_{C \in \bar{C}}{|C|}}
\end{equation}

where $\mathit{cluster}(s)$ is the cluster sentence $s$ was generated from.

We include the normalization by maximum cluster size to squash the score in the general order of magnitude of the other scores and to account for the fact that what is considered ''large'' for a cluster may vary based on the corpus.

%This notion can be very naturally encoded in our framework by considering cluster size. % What about multiple clusters for the same date as a scoring metric?

\subsection{Path Weight}

While we already consider path weight in the generation stage, this does not always guarantee that proposed candidates were generated by a low weight path. If the generation graph is small enough for all or most paths to be generated, then even low weight paths might be included in the candidate set.
To account for this, we introduce a score based on path weight. To align with the notion common in other scores that larger values are preferable, we invert the weight:

\begin{equation}
f_{path}(s) = (\mathit{pathweight}(s) + 1)^{-1}
\end{equation}

where $\mathit{pathweight}$ is the weight of the path that $s$ was generated from (see section \ref{sec:baseline-generation} for details).


\section{Selection using Submodular Functions} \label{sec:system-selection}.

%\subsection{Integer Linear Programming}

%Formulating an Integer Linear Program (ILP) for infering an optimal summary given some sentences and associated importance scores is a common %approach in document summarization. %(CITE!)
%
%Given the set of generated sentences $G$ and the scoring function $f_{score}$ (s. \ref{sec:scoring}) we formulate the following maximization %problem.
%
%\begin{equation}
%\mathit{maximize} \sum_{g \in G} p_g * f_{score}(g)
%\end{equation}
%
%where $p_g$ is a boolean indicator variable indicating whether the generated sentence $g$ should be included into the summary or not.
%
%Following \citet{banerjee}, we also include two sets of constraints into the problem.
%The first constraint ensures non-redundancy by enforcing that of any two sentences with a similarity greather than some threshold $t$ only %one may be included in a summary:
%
%\begin{equation}
%p_g + p_{g'} \leq 1 \forall g \neq g' \land sim(g, g') > t
%\end{equation}
%
%Following \citet{banerjee}, we set $t = 0.8$ and use cosine similarities between vector representations of the two sentences for similarity %computation.
%The vector representations are derived from a tf-idf model fitted on the entire collection of input documents.
%
%The second set of constraints ensures that only one sentence from the same cluster may be included in the final summary:
%
%\begin{equation}
%\sum_{g \in C} p_g \leq 1 \forall C \in \bar{C}
%\end{equation}
%
%Finally, we also need to add a length constraint. We introduce a per-day summary length constraint:
%
%\begin{equation}
%\sum_{g} p_g \leq k
%\end{equation}

As described in section \ref{sec:baseline-selection}, the MDS system of \citet{banerjee} we use as a baseline and starting point for our system, solves an ILP to find an optimal summary from their candidate sentences.
However, as \citet{mcdonald} has shown, running ILP to find a summarization is challenging even for a moderate amount of textual units and quickly becomes infeasible.
This problem is aggravated on TLS corpora, as they can be an order of magnitude larger than traditional MDS systems. For example, while the DUC2004 data has at most 10 documents per input cluster, the Crisis corpus we use for evaluation has $\approx 4000$ news stories per topic. Additionally, generating a huge number of candidate sentences can increase the number of sentences even more, further limiting the usefulness of ILPs. As a consequence the system with the ILP can only be run per date and with external date selection. Furthermore, inter-date redundancy can not be prevented by the constraints.

%\subsection{Submodular Functions}

We thus turn to a computationally more efficient way to select sentences.
\citet{markert} have demonstrated that the objective for timeline summarization can be cast as a monotone, positive submodular function whose optimum can be reasonably well approximated by a greedy algorithm.

The submodularity property is defined as follows:
 Let $G$ be a finite set. A function $f: 2^{G} \rightarrow \mathbb{R}$ is submodular, if it fulfills the diminishing returns property: for $A \subset B \subseteq G$ and some $e \in G \setminus B$ $f(A) \leq f(B)$ implies $f(A \cup e) - f(A) \geq f(B \cup \{e\}) - f(\{B\})$.

A submodular function for TLS is a function $f: 2^{G} \rightarrow \mathbb{R}$, where $G$ is the set of sentences available for summarization.
The objective of the summarization algorithm is then to find a set $T^{*} \subseteq G$ that maximizes $f$.
In fact, a function that just sums up $f_{\mathit{score}}$ over all sentences in $T$ is submodular. This alone, however, is not sufficient, because it does not punish the system for repeating two similar and thus redudant sentences.

To avoid redundancy, we adapt a function from a submodular MDS system by \citet{lin+blimes} designed for this purpose, that was also used by \citet{markert} for their TLS system.

\begin{equation}
f_{\mathit{submod}}(T) = \sum_{i = 1}^{m} \sqrt{\sum_{s \in G \cap P_i} f_{\mathit{score}}(s)}
\end{equation}

where $T \subseteq G$ and $P_1, ..., P_m$ form a semantic partitioning of the generated sentences into $m$ disjoint sets. Note, that these are not the same clusters that were used for candidate generation. Including more than one sentence from the same cluster results in diminishing returns for further sentences from the same cluster, because the square root function will saturate.
By partitioning semantically similar sentences into the same block, this property can be used to avoid redundancy in the summary.
Following \citet{lin+blimes} we obtain the clustering by k-means clustering on the tf-idf vectors of the generated sentences. We do, however, modify the number of clusters we generate. Both \citet{lin+blimes} and \citet{markert} use $0.2 * |U|$ clusters. The analogue for our setting would be $0.2 * |G|$ clusters. This ignores, however, that the generation process naturally produces a much larger number of very similar sentences. We instead choose to set the number of clusters based on the average number of sentences in the daily summaries multiplied with the number of dates to summarize times 2.

We introduce a cluster constraint as mentioned in section \ref{sec:baseline-selection}, to ensure no two sentences from the same cluster are included in the summary:

\begin{equation}
\forall C \in \bar{C} \label{eq:constr-cluster}: |C \cap T| \leq 1
\end{equation}

where $T$ is a timeline and $\bar{C}$ is the set of all clusters.

We also introduce the same length constraints \citet{markert} have used in their system: A maximum number $l$ of dates and a maximum number $k_{\mathit{sent}}$ of sentences per daily summary. 

Formally, for every timeline $T$ we require that
\begin{equation}
|\{ \mathit{date}(s) | s \in T \}| \leq l \label{eq:constr-date}
\end{equation}
and
\begin{equation}
\forall s \in T |\{ s' \in T | \mathit{date}(s) = \mathit{date}(s') \}| \leq k_{\mathit{sent}} \label{eq:constr-len-sent}
\end{equation}

For these constraint \citet{markert} have shown the following lemma:
\begin{lemma}
Under the constraints \ref{eq:constr-len-sent} and \ref{eq:constr-date} the greedy algorithm has a performance guarantee of $\frac{1}{1 + k_{\mathit{sent}}}$.
\end{lemma}

This lemma is, however, not directly applicable to our system, as we also introduce constraint \ref{eq:constr-cluster}. We can show though that this constraint does not have an effect on the guarantee and the following lemma holds (for a proof see \ref{sec:proof-cluster-constr}):
\begin{lemma}
Under the constraints \ref{eq:constr-cluster}, \ref{eq:constr-len-sent} and \ref{eq:constr-date} the greedy algorithm has a performance guarantee of $\frac{1}{1 + k_{\mathit{sent}}}$.
\end{lemma}

As mentioned in section \ref{sec:baseline-selection}, traditional MDS is usually constrained by tokens, not by sentences.
We thus introduce an alternative to constraint \ref{eq:constr-len-sent}:

\begin{equation}
\forall s \in T: \sum_{s' \in T, \mathit{date}(s) = \mathit{date}(s')} \mathit{len}(s') \leq k_{\mathit{tok}} \label{eq:constr-len-tok}
\end{equation}
where $\mathit{len}(s)$ is the length in tokens of sentence $s$ and $k_{\mathit{tok}}$ is the maximum number of tokens allowed per date.

We can show that when this constraint is used the performance guarantee is dependent on both $k_{tok}$ and the minimum number of tokens per sentence $h$ (for a proof, see section \ref{sec:proof-token-constr}):
\begin{lemma}
Under the constraints \ref{eq:constr-cluster}, \ref{eq:constr-date} and \ref{eq:constr-len-tok} the greedy algorithm has a performance guarantee of $\frac{1}{1 + \lfloor k_{\mathit{tok}} / h \rfloor}$.
\end{lemma}

For our setup, we have $h = 8$ due to the way sentences are generated (see section \ref{sec:baseline-generation}).

 %For the usual timeline length constraints, length of daily summaries and number of dates in the timeline, \citet{markert} have shown that a greedy algorithm can solve the problem with a guaranteed lower bound of $1 / (k_s + 1)$ of the optimal value, where $k_s$ is the maximum number of sentences in a daily summary.



%In addition to the 

Note, that with this selection algorithm it is no longer necessary to limit the number of generated sentences to just 200, so we move the limit to 2500 sentences.
We do not completely eliminate the limit, because the huge number of possible paths in a graph makes it infeasible to always generate all simple paths. The cubic complexity of the k-shortest simple paths algorithm further limits the number of sentences we can generate in an acceptable timeframe.

\chapter{Corpora and Experimental Setup} \label{sec:setup}

\section{Corpora} \label{sec:setup-corpora}

We use two different timeline corpora for all our experiments, the \textit{Timeline 17} corpus \citep{tran-tl17} and the \textit{Crisis} corpus \citep{tran-headlines}. They both have a similar structure. Each covers a set of different events and has one or more timelines per event that were crawled from large news websites. The timelines all have different lengths and often cover different periods of time. The corpus also includes a set of news documents that were crawled from news articles relating to the timeline. All articles were stripped of HTML tags and annotated with the date they were published, the document creation time.

While the \textit{Crisis} dataset focuses on several uprisings resulting from the Arab Spring, the Timeline 17 dataset covers a more diverse range of topics from natural disasters to the death of Michael Jackson and subsequent legal proceedings. See table \ref{tab:corpus-topics} for an overview over the events covered in the two corpora.

In table \ref{tab:corpus-stats} we give a brief overview over the characteristics of the two corpora. It is apparent that both corpora are very different in size and distribution of articles. Even though the number of topics is much higher for time Timeline 17 corpus, the Crisis corpus has both more documents and timelines. This is a result of the Timeline 17 corpus having just one timeline for most topics.

\begin{table}
\begin{centering}
\makebox[\textwidth][c]{

\begin{tabular}{|l|ccc|}
\hline
Corpus &TLs & num. docs & num. sents \\\hline
Timeline 17 & 19 & 4,622 & 273,432 \\
Crisis & 22 & 18,246 & 689,165 \\\hline

\end{tabular}
}

\end{centering}

\caption{Corpus Statistics for our evaluation corpora}
\label{tab:corpus-stats}
\end{table}

\begin{table}
\begin{centering}
\makebox[\textwidth][c]{
\begin{tabular}{|l|l|}
\hline
\multirow{8}{*}{\textbf{Timeline 17}} & Deep Water Horizon Oil Spill \\
& Egyptian Revolution \\
& Syrian Civil War \\
& Iraq War \\
& Libyan Civil War \\
& Death of Michael Jackson \\
& Haiti Earthquake \\
& H1N1 Outbreak \\\hline\hline
\multirow{4}{*}{\textbf{Crisis}} & Egyptian Revolution \\
& Crisis in Yemen \\
& Syrian Civil War \\
& Libyan Civil War \\\hline
\end{tabular}

}
\end{centering}

\caption{Topics of our evaluation corpora}
\label{tab:corpus-topics}
\end{table}

\section{Preprocessing}

Our approach requires all input documents to be annotated with part-of-speech tags, which we obtain by running the POS-tagger included in the Stanford CoreNLP suite of NLP tools \citep{stanford-corenlp}. The suite also takes care of proper tokenization and sentence splitting for us.
To obtain sentence level annotation of date expressions, we run the HeidelTime tagger \citep{heideltime}. Heideltime is a rule based system that tags date expressions in a corpus and labels them according to the Timex3\footnote{\url{http://www.timeml.org/}} annotation standard.

\section{Experimental Setting} \label{sec:setup-settings}

As \citet{markert} note, there are a number of different settings for TLS.
For example, \citet{chieu} create summaries with a fixed number of dates and one sentence summaries, while \citet{yan-trans} use a fixed compression ratio.
We follow the evaluation setup \citeauthor{markert} propose and parameterize our timelines by two properties: Number of dates and number of sentences per date.

While restricting the length in terms of number of sentences is a common setting in TLS, used for example by \citet{chieu} and \citet{yan-trans}, traditional summarization is usually restricted by number of tokens, see for example the DUC 2005 challenge \citep{duc2005}. In fact, restricting the number of sentences enables our system to effectively circumvent the length constraints by fusing several sentences together, to form one very long one.
We thus also evaluate each system using the constraints for the number of tokens in a summary we have described in sections \ref{sec:baseline-selection} and \ref{sec:system-selection}.

As timelines also differ in length and covered time periods, we follow \citet{markert} and generate one system timeline per reference timeline with the appropriate parameters. The desired date range of the output is set from the first and last date of the timeline. The number of dates allowed is the number of dates in the reference timeline and the per-summary length constraint is determined by averaging the number of sentences or tokens per daily summary in the reference.

We run the clustering and sentence generation only once for every topic and afterwards restrict the generated clusters to the appropriate date range for each timeline.

\subsection{Oracle Summaries}

In addition to the regular evaluation, we also produce a set of oracle summaries. We compute the ROUGE-2 evaluation score of each candidate sentence, instead of using our scoring methods. For a description of the score see section \ref{tab:evaluation-auto-setup}. The ROUGE-score is computed with respect to the daily summary of the cluster date of the sentence. We also add a small value $\epsilon$ to each score, to avoid sentences with a score of 0. This accounts for the fact that some sentences might have an incorrect date tag, but can still be useful, because their deviation from the correct date is small. Without this small change, our selection functions would just discard the sentences outright.

The rest of the system remains unchanged. Note, that the resulting summaries are not true oracle summaries, in that they are not truly optimal solutions. They merely show the performance our system could have with perfect information about the score of each sentence. The selection phase still uses the ILP or submodular sentence selection methods. The oracles also ignore the grammaticality of included sentences. They are, however, a useful tool that can help evaluate both the effect of summarizing from abstractive candidate sentences instead of the original corpus and the quality of our clustering methods.



\section{Corpus Statistics} \label{sec:setup-stats}

Before we evaluate our systems, we first gather some statistics about the properties of our corpus.
One characteristic we have repeatedly emphasized is the importance of date references.
To quantify this, we count the number of sentences in each corpus, that have an exact reference to a date that is not the publishing date.
We also count the number of times the ten most frequent dates in every topic are referred to in documents published at another date.
In table \ref{tab:datestats} we show the two statistics for both our corpora.

\begin{table}
\begin{centering}
\begin{tabular}{|l|cc|}
\hline
Corpus & Num. sent. with outside reference & Avg. num. docs@10 \\\hline
\textbf{Timeline 17} & 15,300 & 24.1 \\
\textbf{Crisis} & 44,971 & 94.6 \\\hline
\end{tabular}
\end{centering}

\caption{Date reference statistics of our evaluation corpora. The left column is the number of sentences that mention a date that is not their publishing date. The right column is the number of documents the ten most frequent dates are mentioned in, that were published at different dates.}
\label{tab:datestats}
\end{table}

The statistics confirm our assumption that important dates are mentioned across many documents in the corpus. It also shows that there are many sentences with references to dates that are not the same as their publishing dates. This highlights the potential benefits of tagging sentences individually and allowing sentences that are part of documents published at different dates to be clustered together.

We also evaluate the characteristics of our three clustering approaches on the two corpora.
Table \ref{tab:genstats} shows the average number of clusters and the average number of generated candidates for our two evaluation corpora.

\begin{table}
\begin{centering}
\begin{tabular}{|l|cc|}
\hline
System & Avg. num. of clusters & Avg. num. of candidates \\\hline
\multicolumn{3}{|l|}{\textbf{Timeline 17}}\\\hline
Affinity Propagation & 3,816.3 & 233,330.6 \\
Agglomerative  & 224.7 & 46,198.8 \\
Document (Baseline) & 530.1 & 2,368.3 \\\hline
\multicolumn{3}{|l|}{\textbf{Crisis}}\\\hline
Affinity Propagation & 21,094.8 & 1,615,968.2 \\
Agglomerative & 1,060.5 & 298,811.5 \\
Document (Baseline) & 4,872.3 & 21,453.0 \\\hline
\end{tabular}
\end{centering}
\caption{Average number of clusters and generated sentences by clustering methods per topic}
\label{tab:genstats}
\end{table}

We can see that our two newly developed clustering algorithms generate more clusters and thus also more candidates than the document based clustering of our baseline. Especially the difference between the affinity propagation algorithm and the other two systems is very noticeable.
Another interesting observation is that while agglomerative clustering produces fewer clusters than document based clustering, it generates more candidates. This suggests that clusters on the baseline lack the ability to create graphs from which many new sentences can be generated.

The results also show the difference in scale between extractive summarization and our approach to abstractive summarization.
While topics have an average of $150,438.5$ sentences for the Crisis corpus, the number of candidates is ten times that.
The picture is similar on the Timeline 17 corpus, where an average of $27,218.7$ sentences per topic, turns into $233,330.6$ generated candidates. Note though, that not all sentences and candidates will be involved in the generation of every timeline, as some of the dates fall outside their covered date range.
Still, these statistics indicate that our abstractive system vastly increases the number of sentences available for summarization compared to what would be available to an extractive system.


%\section{System Configurations}
%
%In addition to the baseline system, we test our two alternative clustering algorithms, the agglomerative multiclustering (AG) and the affinity propagation variant (%AP).
%We also combine 
%

\chapter{Automatic Evaluation} \label{sec:results-auto}

\section{Evaluation Measures} \label{sec:evaluation-auto-setup}

We evaluate the performance of our systems using the evaluation framework of \citet{tilse}, which is based on the ROUGE \citep{rouge} metric.

Specifically, we use ROUGE-1 and ROUGE-2 metrics.
ROUGE is based on n-gram overlap between a system summary and a set of gold-standard reference summaries.

Let $\mathit{ng}(X)$ be the set of all N-grams in a summary $X$ and let $\mathit{cnt}_{X}(g)$ be the number of times the N-Gram $g$ occurs in $X$.
For two summaries $X_1, X_2$ we define $\mathit{cnt}_{X_1, X_2}(g) = \min\{  \mathit{cnt}_{X_1}(g),  \mathit{cnt}_{X_2}(g) \}$.

ROUGE-N recall is defined as

\begin{equation}
    \mathtt{ROUGE-N-R}(\bar{R}, T) =  \frac{
        \sum_{R \in \bar{R}}
        \sum_{g \in \mathit{ng}(R)}
        \mathit{cnt}_{R,T}(g)
    } {
        \sum_{R \in \bar{R}}
        \sum_{g \in \mathit{ng}(R)}
        \mathit{cnt}_{R(g)}
    }
\end{equation}

and ROUGE-N precision is defined as

\begin{equation}
    \mathtt{ROUGE-N-P}(\bar{R}, T) =  \frac{\sum_{R \in \bar{R}} \sum_{g \in \mathit{ng}(R)} \mathit{cnt}_{R,T}(g)}{|\bar{R}| * \sum_{R \in \bar{R}} \sum_{g \in \mathit{ng}(R)} \mathit{cnt}_{R}(g)}
\end{equation}

where $\bar{R}$ is a set of reference summaries and $T$ is a system summary.
Based on recall and precision we can define the ROUGE-F1 score:

\begin{equation}
    \mathtt{ROUGE-N-F1}(\bar{R}, T) = 2 * \frac{\mathtt{ROUGE-N-R}(\bar{R}, T) * \mathtt{ROUGE-N-P}(\bar{R}, T)}{\mathtt{ROUGE-N-R}(\bar{R}, T) + \mathtt{ROUGE-N-P}(\bar{R}, T)}
\end{equation}

We use report scores from three different scoring methods offered by \citet{tilse}:
\begin{description}
\item[concat] {is obtained by concatenating all date summaries in the timelines. The ROUGE metric can then be computed on the pair just like in traditional summarization.
This completely ignores the partition of the timeline into daily summaries.}
\item[agree] {is obtained by comparing summaries that have the same date in both reference and system output. While this respects the structure of the timeline, it punishes small deviations in dates overly harsh. If a summary has a wrong date, even if it is just off by one day, it receives a score of $0$.}
\item[align m:1] {solves this problem by allowing daily summaries of the system timeline to be aligned with summaries from the reference timelines, even if their dates do not match. The measure computes an alignment $f$ between the dates in the references and those in the summary based on content overlap and the distance of the dates. The ROUGE score is computed by comparing the summaries for the dates aligned by $f$.
The final score is discounted according to the time difference in days of aligned dates. For ROUGE-N the recall is then defined as:

\begin{equation}
    \mathtt{ROUGE-N-R}(\bar{R}, T, d) =  \frac{
        \sum_{d \in E_{\bar{R}}} t_{d, f(d)}
        \sum_{R \in \bar{R}(d)}
        \sum_{g \in \mathit{ng}(R)}
        \mathit{cnt}_{R,T(d)}(g)
    }{
        \sum_{d \in E_{\bar{R}}}
        \sum_{R \in \bar{R}}
        \sum_{g \in \mathit{ng}(R)}
        \mathit{cnt}_R(g)
    }
\end{equation}
where $E_{\bar{R}}$ is the set of all dates in the reference summaries,
$f(d)$ is the date that the reference date $d$ was aligned to and $T(f(d))$ is the possibly empty system summary of that date.
$t_{d,f(d)}$ is a penalty term for aligning dates that are far apart.
It is defined as
\begin{equation}
t_{d_R, d_T} = \frac{1}{|d_R - d_T| + 1}
\end{equation}
in the evaluation framework.


 %metric aligns summaries for different dates based on the time difference between the tag of the reference timeline summaries and the contents of the daily summaries. The final summary score is the ROUGE score between aligned dates discounted proportionally to the time difference between their date tags.
 }
\end{description}


%Note that we compute the per-topic macro average of all metrics, not the average across all reference timelines.
The final scores for a corpus are the average scores over all timelines.

We check for significant differences in the results using an approximate randomization test \citep{art} using a library provided by Sebastian Martschat \footnote{\url{https://github.com/smartschat/art}}. %We find that all differences are significant, except for the improvements in ROUGE-1 align and agree from 

\section{Results}

\subsection{Oracle Summarizations}

We run the oracle summarizer with each of our three clustering methods (document-based clustering, agglomerative clustering and affinity propagation), as well as in an extractive setting, where we select from the sentences in the original documents.
The document clustering is evaluated with the ILP sentence selection method, including prior date selection.%, and with the submodular selection method without prior date selection.
All other systems are run with the submodular selection method only.
The results for the sentence constraint setting are shown in table \ref{tab:results-oracle-sent}, the results for the token constraint setting in table \ref{tab:results-oracle-tok}.

The first interesting finding is that the choice of length constraint does not seem to have a big influence on the results of the oracle summaries. Even where the differences are statistically significant, they are usually small.
We also identify the same trends in both settings, so we focus on the results for the sentence constraint.

%We find that the document clustering method of the baseline performs far worse than the other two clustering methods. Manual analysis reveals that this is in part due to the pipelined date selection. As the dates are selected before the system is run, it is possible that the system does not find a
In general, the best results among the clustering methods are consistently achieved by the affinity propagation based system (AP), followed by the agglomerative multi-clustering system. All improvements of the former over the latter are statistically significant ($p < 0.05$). The baseline system with ILP performs by far worst of all three clustering methods. When looking at the timelines produced by the system, we find that almost $79\%$ of dates have empty summaries. These dates were selected by the date selection algorithm, but the system did not produce an abstractive sentence for them. This points to a general problem with prior date selection: The system is not guaranteed to have a sentence to fill the summary with content. This is exacerbated by the fact, that the timelines have different date ranges than the publishing dates of the articles, so for some dates there are no published documents. For example, the articles in the Syrian part of the Crisis dataset start from 2011/11/17, while the earliest date in any of the timelines is eight months earlier (2011/03/15). As the document clustering method assigns each cluster the document creation time as a date, no candidates can be generated for dates before 2011/11/17. In total, circa $16\%$ of dates in the reference timelines do not have a single document published on them.

To confirm our intuition that prior date selection is indeed problematic, we have also evaluated the document-based clustering method with the submodular sentence selector, which allows us to remove the prior date selection. The results of this experiment, which can also be seen in table \ref{tab:results-oracle-tok}, show that this change leads to a significant improvement in all metrics, except date selection F1.
This confirms our hypothesis that the baseline system is hampered by the need for prior date selection, which does not take into account that some important dates may be difficult for the summarizer to create a summary for.



The affinity propagation system also improves significantly over the extractive oracle in many metrics, though we should emphasize again that these results are specific to our sentence selection method and are not true optimal summaries. One should also keep in mind, that the oracles optimize align-ROUGE-2 score and ignore ROUGE-1 score. It is thus very well possible for a system, even with the same selection method, to outperform these oracles in one of the other metrics. Still, however, these results demonstrate that at least in our setup we can improve the potential summarization scores by moving from purely extractive summarization to abstractive systems.

\begin{table}
\begin{centering}
\makebox[\textwidth][c]{

\begin{tabular}{|l|l|ll|ll|ll|}
\hline

System & Date & \multicolumn{ 2 }{|c|}{Concat} & \multicolumn{ 2 }{c|}{Agree} & \multicolumn{ 2 }{c|}{Align} \\
 & F1 & R1 & R2 & R1 & R2 & R1 & R2 \\\hline\hline

\multicolumn{8}{|c|}{\textbf{Timeline 17}}\\\hline

Affinity Propagation & 0.619$^{\dagger}$&\textbf{0.403}$^{\dagger}$&\textbf{0.163}$^{\dagger\ddagger}$&\textbf{0.206}$^{\dagger}$&\textbf{0.126}$^{\dagger\ddagger}$&\textbf{0.211}$^{\dagger}$&\textbf{0.127}$^{\dagger\ddagger}$\\
Agglomerative & 0.513$^{*}$&0.289$^{*}$&0.094$^{*}$&0.117$^{*}$&0.064$^{*}$&0.125$^{*}$&0.066$^{*}$\\
Baseline + Submod & 0.376&0.242$^{\circ}$&0.058$^{\circ}$&0.054$^{\circ}$&0.019$^{\circ}$&0.067$^{\circ}$&0.021$^{\circ}$\\
Baseline + ILP & 0.526$^{\circ}$&0.101&0.026&0.017&0.005&0.018&0.006\\
Extractive & \textbf{0.630}&0.374&0.120&0.157&0.089&0.163&0.090\\\hline
\hline\multicolumn{8}{|c|}{\textbf{Crisis}}\\\hline
Affinity Propagation & \textbf{0.630}$^{\dagger}$&\textbf{0.437}$^{\dagger\ddagger}$&\textbf{0.185}$^{\dagger\ddagger}$&\textbf{0.254}$^{\dagger\ddagger}$&\textbf{0.168}$^{\dagger\ddagger}$&\textbf{0.260}$^{\dagger\ddagger}$&\textbf{0.169}$^{\dagger\ddagger}$\\
Agglomerative & 0.401$^{*}$&0.335$^{*}$&0.105$^{*}$&0.141$^{*}$&0.087$^{*}$&0.151$^{*}$&0.088$^{*}$\\
Baseline + Submod  & 0.239&0.300$^{\circ}$&0.054$^{\circ}$&0.050$^{\circ}$&0.020$^{\circ}$&0.059$^{\circ}$&0.021$^{\circ}$\\
Baseline + ILP & 0.293$^{\circ}$&0.181&0.027&0.016&0.005&0.021&0.005\\\hline
Extractive & 0.588&0.378&0.126&0.189&0.112&0.194&0.112\\\hline
\end{tabular}

}
\end{centering}

\caption{Results of the oracle summarizer with sentence constraint. $^{\dagger}$ indicates statistically significant improvement ($p < 0.05$) of affinity propagation over agglomerative clustering. $^\ddagger$ indicates significant improvement of affinity propagation over the extractive system. $^*$ indicates significant improvement of agglomerative clustering over the baseline with submodular selection.  $^\circ$ indicates statistically significant improvement of submodular baseline over ILP baseline or vice versa.}
\label{tab:results-oracle-sent}
\end{table}

\begin{table}
\begin{centering}
\makebox[\textwidth][c]{

\begin{tabular}{|l|l|ll|ll|ll|}
\hline
System & Date & \multicolumn{ 2 }{c|}{Concat} & \multicolumn{ 2 }{c|}{Agree} & \multicolumn{ 2 }{c|}{Align} \\
 & F1 & R1 & R2 & R1 & R2 & R1 & R2 \\\hline\hline

\multicolumn{8}{|c|}{\textbf{Timeline 17}}\\\hline
Affinity Propagation & 0.610$^{\dagger}$&\textbf{0.413}$^{\dagger}$&\textbf{0.157}$^{\dagger-}$&\textbf{0.195}$^{\dagger-}$&0.114$^{\dagger-}$&\textbf{0.201}$^{\dagger-}$&0.115$^{\dagger-}$\\
Agglomerative & 0.531$^{*}$&0.302$^{*+}$&0.094$^{*}$&0.118$^{*}$&0.063$^{*}$&0.126$^{*}$&0.065$^{*}$\\
Baseline + ILP & 0.523$^{\circ}$&0.101&0.026&0.017&0.005&0.018&0.006\\
Baseline + Submod & 0.365&0.247$^{\circ}$&0.059$^{\circ}$&0.051$^{\circ}$&0.019$^{\circ}$&0.064$^{\circ}$&0.021$^{\circ}$\\\hline
Extractive & \textbf{0.665}&0.407&0.155&0.190&\textbf{0.123}&0.196&\textbf{0.124}\\\hline\\\hline
\hline\multicolumn{8}{|c|}{\textbf{Crisis}}\\\hline
Affinity Propagation  & \textbf{0.606}$^{\dagger}$&\textbf{0.440}$^{\dagger\ddagger}$&\textbf{0.177}$^{\dagger\ddagger-}$&\textbf{0.240}$^{\dagger\ddagger-}$&\textbf{0.155}$^{\dagger\ddagger-}$&\textbf{0.246}$^{\dagger\ddagger-}$&\textbf{0.156}$^{\dagger\ddagger-}$\\
Agglomerative & 0.411$^{*}$&0.348$^{*}$&0.103$^{*}$&0.137$^{*}$&0.080$^{*-}$&0.146$^{*}$&0.081$^{*-}$\\
Baseline + ILP   & 0.296&0.183&0.027&0.017&0.005&0.022&0.006\\
Baseline + Submod &  0.259&0.311$^{\circ+}$&0.054$^{\circ}$&0.051$^{\circ}$&0.020$^{\circ}$&0.060$^{\circ}$&0.021$^{\circ}$\\\hline
Extractive & 0.542&0.359&0.113&0.170&0.100&0.175&0.101\\\hline
\end{tabular}

}
\end{centering}


\caption{Results of the oracle summarizer with token constraint. $^+$ and $^-$ indicate statistically significant increase or decrease in score ($p < 0.05$) from the same system under sentence constraint. The other symbols are the same as in table \ref{tab:eval-cl-tr}}
\label{tab:results-oracle-tok}

\end{table}


%Compared to the extractive baseline, the AP system performs statistically significantly worse than the extractive baseline in every metric.
%Interestingly this is not true in the case of the Crisis dataset. Here the abstractive oracle performs better in every aspect except ROUGE-1 concat and date selection. Differences are significant except for ROUGE-1 align for the token constraint and ROUGE-1 agree for both sentence and token constraint.

% Has been more difficult in the past -> so what?

\subsection{Sentence Constraint Setting}

%Table \ref{tab:rouge-sentence} shows the results for sentence constraints.
%
%All our systems outperform the baseline system by a large margin. This is %unsurprising given the results of the oracle evaluations presented in %section \ref{sec:results-oracle}. Note that the relatively good date %selection must be attributed to the dedicated date selection algorithm, %not the summarization component itself.
%
%We achieve the best ROUGE-2 scores across all three metrics using the %affinity propagation method for clustering and a combination of all our %features for scoring on both corpora.
%Interestingly there is a huge jump in performance between the system %including path weight and 
%the one not including it. This comes at a cost of decreased ROUGE-1 score %on the Crisis dataset, although the ROUGE score is even higher with path %weight on the timeline17 dataset. This is unsurprising, if we consider %that the other content scoring method, the TextRank scores, are computed %per keyword and do not carry any information on the phrase level.



We now investigate the clustering methods in a practical context, along with our modified TextRank functions.
We evaluate our agglomerative and affinity propagation clustering methods, along with the TextRank variants we have developed to score informativeness. Sentence selection is conducted with the submodular method.
The scoring functions all include the linguistic quality score and one of our text rank scores, so we use the shorthand $TR^{(date)}$ where we mean the scoring function $f_{TR^{(date)} * ling}$, for example.

For comparison, we also run our baseline system, which means prior date selection, document based clustering, per-cluster informativeness score and ILP-based sentences selection.
In addition to that, we also run the baseline system with the submodular sentence selection method, following our experience with the oracle summaries.
The results are shown in table \ref{tab:eval-cl-tr}.

On the Timeline 17 dataset we find that while both our clustering algorithms outperform the document-clustering baseline in most cases, the affinity propagation clustering achieves better results than the agglomerative multiclustering system. This confirms the results of the oracle summarization, where the affinity propagation oracle outperformed the agglomerative clustering based system considerably.

Of the four informativeness scores we have applied to our problem, the per-cluster score generally performs worst on the Timeline 17 dataset, while the $TR^{(temp)}$ score has the best results, although most differences are not statistically significant.

On the Crisis dataset, results are less clear.
Scores are very low, with almost every system taking the lead in at least one category.
A surprising result is that the submodular variant of the baseline system, which performs consistently worse than the other two clustering methods on the Timeline 17 corpus, performs better than any other system on the Crisis corpus in the agree and align metrics.

We can explain the difference in behavior of this clustering method on the two corpora with their different sizes and the different number of candidates generated for them (see table \ref{tab:genstats}).
On the relatively smaller article collections of the Timeline 17 corpus, the document based clustering method has trouble generating enough clusters for the timeline.
On average it generates 17.7 daily summaries fewer per timeline than allowed by the constraints. At the same time, it only generates an average of 1.7 fewer summaries per date on the Crisis corpus.

We speculate that the advantage of the document-based clustering over the other methods on the latter corpus is a result of it depending on the number of documents published at the same date and thus implicitly on the \textit{burstiness} of the dates. It also contains a filter for cluster size directly in the clustering method, as small clusters are deleted. This might give an advantage to the document-based clustering method over the other methods, which do not encode this information.
While we have no direct way of confirming this hypothesis, if it is true, we would expect the document clustering approach to not benefit greatly from the inclusion of additional features that encode cluster size and date importance.


%As date selection is more difficult on the Crisis corpus \citep{markert}, this would also explain the differences between scores on the Timeline 17 and Crisis datasets.


\begin{table}
\begin{centering}
\makebox[\textwidth][c]{

   \cellspacebottomlimit=3pt
   \cellspacetoplimit=3pt

\begin{tabular}{|Sl|Sl|SlSl|SlSl|SlSl|}
\hline

System & Date & \multicolumn{ 2 }{c|}{Concat} & \multicolumn{ 2 }{c|}{Agree} & \multicolumn{ 2 }{c|}{Align} \\
 & F1 & R1 & R2 & R1 & R2 & R1 & R2 \\\hline\hline
\multicolumn{8}{|c|}{ \textbf{Timeline 17}}\\\hline
\multicolumn{8}{|Sl|}{ \textit{Agglomerative Clustering} }\\\hline
$TR^{(date)}$  & 0.418    & 0.307 & 0.058 & 0.055 & 0.014 & 0.066 & 0.015 \\
$TR^{(global)}$    & 0.434    & 0.315 & 0.062 & 0.055 & 0.013 & 0.067 & 0.014 \\
$TR^{(cluster)}$     & 0.438$^{\ddagger}$   & 0.308$^{\ddagger}$    & 0.060 & 0.059$^{\ddagger}$    & 0.016 & 0.069 & 0.018 \\
$TR^{(temp)}$   & 0.430    & 0.313 & 0.060 & 0.058 & 0.016$^{a}$   & 0.069 & 0.017$^{a}$ \\\hline
\multicolumn{8}{|Sl|}{ \textit{Affinity Propagation} }\\\hline
$TR^{(date)}$   & 0.434    & 0.345 & 0.073$^{*}$   & 0.068$^{*}$   & 0.016 & 0.079$^{*}$   & 0.018 \\
$TR^{(global)}$     & 0.477    & 0.347 & 0.069 & 0.067$^{*}$   & 0.015 & 0.079$^{*}$   & 0.016 \\
$TR^{(cluster)}$     & 0.473$^{\dagger}$    & 0.338$^{\dagger}$ & 0.068$^{\dagger}$ & 0.063$^{\dagger}$ & 0.014 & 0.073$^{\dagger}$ & 0.016 \\
$TR^{(temp)}$   & 0.448    & \textbf{0.356}$^{a}$  & \textbf{0.075}$^{*}$  & \textbf{0.071}    & \textbf{0.018}    & \textbf{0.084}$^{*}$  & \textbf{0.020}$^{b}$ \\\hline
\multicolumn{8}{|Sl|}{ \textit{Document-Guided Clustering (Baseline)} }\\\hline
$TR^{(cluster)} + Submod$      & 0.358    & 0.258$^{\circ}$   & 0.051$^{\circ}$   & 0.042$^{\circ}$   & 0.011$^{\circ}$   & 0.055$^{\circ}$   & 0.012$^{\circ}$ \\
$TR^{(cluster)}$ + ILP (MDS)     & \textbf{0.521}$^{\circ}$ & 0.131 & 0.026 & 0.018 & 0.004 & 0.021 & 0.004 \\\hline
\hline\multicolumn{8}{|c|}{ \textbf{Crisis}}\\\hline
\multicolumn{8}{|Sl|}{ \textit{Agglomerative Clustering} }\\\hline
$TR^{(date)}$   & 0.100$^{*}$  & 0.325 & 0.049 & 0.015$^{*}$   & 0.004$^{*}$   & 0.028$^{*}$   & 0.005$^{*}$ \\
$TR^{(global)}$     & 0.122    & 0.319$^{*}$   & 0.046 & 0.017 & 0.003 & 0.033$^{b}$   & 0.004 \\
$TR^{(cluster)}$     & 0.162    & 0.320 & 0.048 & 0.023$^{*}$   & 0.005$^{*}$   & 0.038$^{*}$   & 0.007$^{*}$ \\
$TR^{(temp)}$   & 0.116    & 0.323 & 0.046 & 0.016 & 0.003 & 0.028 & 0.005 \\\hline
\multicolumn{8}{|Sl|}{ \textit{Affinity Propagation} }\\\hline
$TR^{(date)}$   & 0.063    & \textbf{0.329}    & 0.050 & 0.007 & 0.001 & 0.019 & 0.003 \\
$TR^{(global)}$     & 0.206$^{*}$  & 0.299 & 0.049 & 0.026$^{*}$   & 0.004 & 0.040$^{*}$   & 0.005 \\
$TR^{(cluster)}$     & 0.181    & 0.327 & 0.045 & 0.016 & 0.002 & 0.030 & 0.003 \\
$TR^{(temp)}$   & 0.125$^{a}$  & 0.324$^{b}$   & \textbf{0.057}$^{*}$  & 0.015$^{a}$   & 0.002 & 0.032$^{a}$   & 0.004$^{a}$ \\\hline
\multicolumn{8}{|Sl|}{ \textit{Document-Guided Clustering (Baseline)} }\\\hline
$TR^{(cluster)}$ + Submod      & 0.206$^{\ddagger}$   & 0.323$^{\circ}$   & 0.046$^{\circ}$   & \textbf{0.032}$^{\dagger\circ}$   & \textbf{0.007}$^{\dagger\circ}$   & \textbf{0.042}$^{\dagger\circ}$   & \textbf{0.008}$^{\dagger\circ}$ \\
$TR^{(cluster)}$ + ILP (MDS)     & \textbf{0.293}$^{\circ}$ & 0.210 & 0.025 & 0.016 & 0.003 & 0.021 & 0.003 \\\hline


\end{tabular}
}
\end{centering}
\caption{Evaluation results for different clustering methods and TextRank scores.
 $\dagger$ indicates significant improvement of affinity propagation clustering with $TR^{(cluster)}$ over document clustering with $TR^{(cluster)}$ or vice versa $(p < 0.05)$, $\ddagger$ is the same for agglomerative clustering. $^*$ marks significant improvement of a scoring method with affinity propagation over its counterpart in the agglomerative clustering system (or vice versa).
$^\circ$ indicates significant improvement of submodular baseline over ILP baseline (or vice versa).
}
\label{tab:eval-cl-tr}
\end{table}

We now investigate the effects of adding additional features to our systems.
Taking into account the scores of the first round of evaluation, we focus our investigation on the affinity propagation clustering method and the global, local and date TextRank scores to evaluate the effect of adding the additional scoring functions we described in section \ref{sec:scoring}.
We compare both the effect of adding only the $f_{dateref}$ and $f_{clsize}$ metrics, as well as also adding the $f_{path}$ score in addition to that. We also evaluate the submodular variant of the baseline system, on account of its performance on the Crisis dataset.
The results of that evaluation are presented in table \ref{tab:results-features}.

We can observe that the addition of $f_{dateref}$ and $f_{clsize}$ leads to a significant jump in performance in most cases.
Adding the path weight seems to always have a positive effect on the ROUGE-2 scores, while only slightly affecting ROUGE-1 in most cases.
This is to be expected, as the path weight does not only take into account unigram occurrence, but can also assign scores to longer phrases, which the TextRank scores can not.

The baseline performs much worse in comparison this time on the Crisis dataset as well, much more in line with its performance on the Timeline 17 dataset. Apparently it does not benefit from the new information as much as the other systems. This supports our earlier hypothesis that the document-based clustering algorithm indeed contains some information about date importance and cluster size implicitly already, which the other methods lack.

Interestingly, there seems to be a very huge jump in performance in case of the $TR^{(date)}$ informativeness metric, which performs much better than all other systems on the Crisis dataset. We investigate this surprising result and find that we can attribute most of the high score to the timelines generated for only one of the four corpora in the dataset, the Libyan revolution.
If we remove this dataset from our evaluation, results for the system are still slightly better than for the other system, but the gap becomes much smaller and the differences statistically insignificant $(p \geq 0.05)$.

This reveals a problem with this evaluation corpus. While there is a total of 22 timelines in the corpus, there are only four collections of articles that timelines are generated from. While the timelines have different lengths and date ranges, their date ranges overlap. In case of the Crisis Libya dataset for example, $\approx 54\%$ of all documents are inside the range of all timelines and $\approx 75\%$ are in the ranges of all but one timeline. This means the timelines are largely generated from the same documents. As demonstrated in this example, that can disproportionately reward systems that perform very well on one set of articles.
Note that for other corpora in the Crisis dataset, the overlap is far less extreme.
For Yemen, it is at $\approx 35\%$, for Syria at $\approx 20\%$ and the Egypt topic has no documents that were used in the generation of all timelines.

Finally, we also compare our systems with the extractive system of \citet{chieu}, which we have described in section \ref{sec:prior-work}.
We use the results reported by \citet{markert} for comparison. Note that, unlike ours, this system filters the sentences of the documents using a hand-crafted set of query words before summarization.

Our system seems to outperform this baseline, but we lack full access to the evaluation results, so we can not check for statistical significance.

%\begin{table}
%
%    \cellspacebottomlimit=3pt
%    \cellspacetoplimit=3pt
%
%\begin{centering}
%\makebox[\textwidth][c]{
%\begin{tabular}{|Sl|Sl|SlSl|SlSl|SlSl|}
%\hline
%& Date & \multicolumn{ 2 }{c|}{Concat} & \multicolumn{ 2 }{c|}{Agree} & \multicolumn{ 2 }{c|}{Align} \\
% & \multicolumn{1}{c|}{F1} & \multicolumn{1}{c}{R1} & \multicolumn{1}{c|}{R2} & \multicolumn{1}{c}{R1} & \multicolumn{1}{c|}{R2} & \multicolumn{1}{%c}{R1} & \multicolumn{1}{c|}{R2} \\\hline\hline
%\multicolumn{8}{|l|}{ \textbf{Timeline 17}}\\\hline
%$f_{ling*TR^{(date)}*\text{dateref}*\text{path}}$    & \textbf{0.519}   & 0.334 & 0.080 & \textbf{0.079}    & \textbf{0.023}    & \textbf{0.089}%    & \textbf{0.025} \\\hline
%%$TR^{(date)}$ + Feat. + Path    & \textbf{0.519}   & 0.334 & 0.080 & \textbf{0.079}    & \textbf{0.023}    & \textbf{0.089}    & \textbf{0.025} %\\\hline
%
%
%$f_{ling*TR^{(date)}*\text{dateref}}$    & 0.511$^{\dagger}$    & 0.355 & 0.077 & 0.078 & 0.020 & 0.088 & 0.022 \\\hline
%$f_{ling*TR^{(global)}*\text{dateref}*\text{csize}*\text{path}}$     & 0.507    & 0.334 & \textbf{0.084}    & 0.075 & 0.022 & 0.084 & \textbf{0.025%} \\\hline
%$f_{ling*TR^{(global)}*\text{dateref}*\text{csize}}$     & 0.508$^{\dagger}$    & 0.356$^{*}$   & 0.077$^{\dagger}$ & 0.075$^{\dagger}$ & 0.019$^{\%dagger}$ & 0.085$^{\dagger}$ & 0.020$^{\dagger}$ \\\hline
%$f_{ling*TR^{(temp)}*\text{dateref}*\text{csize}*\text{path}}$   & 0.507    & 0.333 & \textbf{0.084}    & 0.075 & \textbf{0.023}    & 0.085 & \%textbf{0.025} \\\hline
%$f_{ling*TR^{(temp)}*\text{dateref}*\text{csize}}$   & 0.511$^{\dagger}$    & \textbf{0.362}$^{*}$  & 0.079 & 0.074 & 0.018 & 0.085 & 0.020 \\\%hline
%
%Baseline + Feats. + Path  & 0.360& 0.259&0.050&0.042&0.010&0.054&0.011 \\\hline
%Baseline + Feats. & 0.359 &  0.213 & 0.042&0.031&0.008&0.042&0.009 \\\hline
%Chieu & 0.287 & 0.295 & 0.074 & 0.045 & 0.016 & 0.070 & 0.019 \\\hline\hline
%
%\hline\multicolumn{8}{|l|}{ \textbf{Crisis}}\\\hline
%$f_{ling*TR^{(date)}*\text{dateref}*\text{path}}$    & 0.318    & 0.342 & \textbf{0.081}$^{*ab}$    & \textbf{0.072}$^{*b}$ & \textbf{0.027}$^{*ab}%$    & \textbf{0.087}$^{*ab}$    & \textbf{0.029}$^{*ab}$ \\\hline
%$f_{ling*TR^{(date)}*\text{dateref}}$    & 0.310$^{\dagger}$    & \textbf{0.349}$^{\dagger}$    & 0.060$^{\dagger}$ & 0.055$^{\dagger}$ & 0.011$^{\%dagger}$ & 0.071$^{\dagger}$ & 0.014$^{\dagger}$ \\\hline
%$f_{ling*TR^{(global)}*\text{dateref}*\text{csize}*\text{path}}$     & 0.309    & 0.338 & 0.072$^{*}$   & 0.056$^{*}$   & 0.015$^{*}$   & 0.069$^{*%}$   & 0.017$^{*}$ \\\hline
%$f_{ling*TR^{(global)}*\text{dateref}*\text{csize}}$     & 0.293$^{\dagger}$    & 0.337$^{\dagger}$ & 0.057$^{\dagger}$ & 0.045$^{\dagger}$ & 0.009%$^{\dagger}$ & 0.060$^{\dagger}$ & 0.011$^{\dagger}$ \\\hline
%$f_{ling*TR^{(temp)}*\text{dateref}*\text{csize}*\text{path}}$   & \textbf{0.320}$^{*}$ & 0.331 & 0.071 & 0.059 & 0.016$^{*}$   & 0.071 & 0.017 \\\%hline
%$f_{ling*TR^{(temp)}*\text{dateref}*\text{csize}}$   & 0.295$^{\dagger}$    & 0.344$^{\dagger}$ & 0.059 & 0.045$^{\dagger}$ & 0.007$^{\dagger}$ & %0.061$^{\dagger}$ & 0.009$^{\dagger}$ \\\hline
%
%Baseline + Feats. + Path & 0.243 & 0.317 & 0.043 & 0.037 & 0.007 & 0.049 & 0.008 \\\hline
%Baseline + Feats. & 0.245 & 0.271 & 0.032 & 0.025 & 0.004 & 0.032 & 0.005 \\\hline
%Chieu & 0.189 & 0.353 & 0.067 & 0.032 & 0.008 & 0.054 & 0.012 \\\hline\hline
%
%\hline\multicolumn{8}{|l|}{ \textbf{Crisis without Libya}}\\\hline
%$f_{ling*TR^{(date)}*\text{dateref}*\text{path}}$    & 0.295    & 0.334 & \textbf{0.074}    & \textbf{0.063}    & \textbf{0.020}    & \textbf{0.078%}    & \textbf{0.022} \\\hline
%$f_{ling*TR^{(date)}*\text{dateref}}$    & 0.310$^{\dagger}$    & \textbf{0.351}$^{\dagger}$    & 0.063$^{\dagger}$ & 0.060$^{\dagger}$ & 0.013$^{\%dagger}$ & 0.076$^{\dagger}$ & 0.015$^{\dagger}$ \\\hline
%$f_{ling*TR^{(global)}*\text{dateref}*\text{csize}*\text{path}}$     & 0.316    & 0.347$^{b}$   & 0.072 & 0.058 & 0.016 & 0.072 & 0.018 \\\hline
%$f_{ling*TR^{(global)}*\text{dateref}*\text{csize}}$     & 0.305$^{\dagger}$    & 0.330 & 0.055$^{*}$   & 0.048$^{\dagger}$ & 0.009 & 0.061$^{\%dagger}$ & 0.010 \\\hline
%$f_{ling*TR^{(temp)}*\text{dateref}*\text{csize}*\text{path}}$   & \textbf{0.331}$^{a}$ & 0.335 & 0.070 & 0.061 & 0.016 & 0.074 & 0.018 \\\hline
%$f_{ling*TR^{(temp)}*\text{dateref}*\text{csize}}$   & 0.310$^{\dagger}$    & 0.344 & 0.059 & 0.049$^{\dagger}$ & 0.007$^{\dagger}$ & 0.064$^{\%dagger}$ & 0.009$^{\dagger}$ \\\hline
%\end{tabular}
%}
%\end{centering}
%
%
%\caption{Evaluation results for affinity propgation with additional features.
%$^\dagger$ indicates significant improvement of a system with features in comparison to its featureless counterpart in table \ref{tab:eval-cl-tr}.
%$^{*}$ indicates significant improvement of a system with path weight scoring over its counterpart without or vice verse.
%$^{a}$ and $^b$ mark significant improvement of $TR^{(date)}$ system with path over $TR^{(temp)}$ with path and $TR^{(global)}$ with path %respectively (or vice versa).
%}
%\label{tab:results-features}
%
%\end{table}



\begin{table}

    \cellspacebottomlimit=3pt
    \cellspacetoplimit=3pt

\begin{centering}
\makebox[\textwidth][c]{
\begin{tabular}{|Sl|Sl|SlSl|SlSl|SlSl|}
\hline
System & Date & \multicolumn{ 2 }{c|}{Concat} & \multicolumn{ 2 }{c|}{Agree} & \multicolumn{ 2 }{c|}{Align} \\
 & \multicolumn{1}{c|}{F1} & \multicolumn{1}{c}{R1} & \multicolumn{1}{c|}{R2} & \multicolumn{1}{c}{R1} & \multicolumn{1}{c|}{R2} & \multicolumn{1}{c}{R1} & \multicolumn{1}{c|}{R2} \\\hline\hline
\multicolumn{8}{|c|}{ \textbf{Timeline 17}}\\\hline
$TR^{(date)}$ + Feat. + Path    & \textbf{0.519}   & 0.334 & 0.080 & \textbf{0.079}    & \textbf{0.023}    & \textbf{0.089}    & \textbf{0.025} \\
$TR^{(date)}$ + Feat.    & 0.511$^{\dagger}$    & 0.355 & 0.077 & 0.078 & 0.020 & 0.088 & 0.022 \\
$TR^{(global)}$ + Feat. + Path & 0.507    & 0.334 & \textbf{0.084}    & 0.075 & 0.022 & 0.084 & \textbf{0.025} \\
$TR^{(global)}$ + Feat.& 0.508$^{\dagger}$    & 0.356$^{*}$   & 0.077$^{\dagger}$ & 0.075$^{\dagger}$ & 0.019$^{\dagger}$ & 0.085$^{\dagger}$ & 0.020$^{\dagger}$ \\
$TR^{(temp)}$ + Feat. + Path  & 0.507    & 0.333 & \textbf{0.084}    & 0.075 & \textbf{0.023}    & 0.085 & \textbf{0.025} \\
$TR^{(temp)}$ + Feat.   & 0.511$^{\dagger}$    & \textbf{0.362}$^{*}$  & 0.079 & 0.074 & 0.018 & 0.085 & 0.020 \\\hline

Baseline + Feat. + Path  & 0.360& 0.259&0.050&0.042&0.010&0.054&0.011 \\
Baseline + Feat. & 0.359 &  0.213 & 0.042&0.031&0.008&0.042&0.009 \\\hline
Chieu & 0.287 & 0.295 & 0.074 & 0.045 & 0.016 & 0.070 & 0.019 \\\hline\hline

\hline\multicolumn{8}{|c|}{ \textbf{Crisis}}\\\hline
$TR^{(date)}$ + Feat. + Path    & 0.318    & 0.342 & \textbf{0.081}$^{*ab}$    & \textbf{0.072}$^{*b}$ & \textbf{0.027}$^{*ab}$    & \textbf{0.087}$^{*ab}$    & \textbf{0.029}$^{*ab}$ \\
$TR^{(date)}$ + Feat. & 0.310$^{\dagger}$    & \textbf{0.349}$^{\dagger}$    & 0.060$^{\dagger}$ & 0.055$^{\dagger}$ & 0.011$^{\dagger}$ & 0.071$^{\dagger}$ & 0.014$^{\dagger}$ \\
$TR^{(global)}$ + Feat. + Path    & 0.309    & 0.338 & 0.072$^{*}$   & 0.056$^{*}$   & 0.015$^{*}$   & 0.069$^{*}$   & 0.017$^{*}$ \\
$TR^{(global)}$ + Feat.  & 0.293$^{\dagger}$    & 0.337$^{\dagger}$ & 0.057$^{\dagger}$ & 0.045$^{\dagger}$ & 0.009$^{\dagger}$ & 0.060$^{\dagger}$ & 0.011$^{\dagger}$ \\
$TR^{(temp)}$ + Feat. + Path  & \textbf{0.320}$^{*}$ & 0.331 & 0.071 & 0.059 & 0.016$^{*}$   & 0.071 & 0.017 \\
$TR^{(temp)}$ + Feat.   & 0.295$^{\dagger}$    & 0.344$^{\dagger}$ & 0.059 & 0.045$^{\dagger}$ & 0.007$^{\dagger}$ & 0.061$^{\dagger}$ & 0.009$^{\dagger}$ \\\hline

Baseline + Feats. + Path & 0.243 & 0.317 & 0.043 & 0.037 & 0.007 & 0.049 & 0.008 \\
Baseline + Feats. & 0.245 & 0.271 & 0.032 & 0.025 & 0.004 & 0.032 & 0.005 \\\hline
Chieu & 0.189 & 0.353 & 0.067 & 0.032 & 0.008 & 0.054 & 0.012 \\\hline\hline

\hline\multicolumn{8}{|c|}{ \textbf{Crisis without Libya}}\\\hline
$TR^{(date)}$ + Feat. + Path    & 0.295    & 0.334 & \textbf{0.074}    & \textbf{0.063}    & \textbf{0.020}    & \textbf{0.078}    & \textbf{0.022} \\
$TR^{(date)}$ + Feat.    & 0.310$^{\dagger}$    & \textbf{0.351}$^{\dagger}$    & 0.063$^{\dagger}$ & 0.060$^{\dagger}$ & 0.013$^{\dagger}$ & 0.076$^{\dagger}$ & 0.015$^{\dagger}$ \\
$TR^{(global)}$ + Feat. + Path    & 0.316    & 0.347$^{b}$   & 0.072 & 0.058 & 0.016 & 0.072 & 0.018 \\
$TR^{(global)}$ + Feat. & 0.305$^{\dagger}$    & 0.330 & 0.055$^{*}$   & 0.048$^{\dagger}$ & 0.009 & 0.061$^{\dagger}$ & 0.010 \\
$TR^{(temp)}$ + Feat. + Path    & \textbf{0.331}$^{a}$ & 0.335 & 0.070 & 0.061 & 0.016 & 0.074 & 0.018 \\
$TR^{(temp)}$ + Feat.   & 0.310$^{\dagger}$    & 0.344 & 0.059 & 0.049$^{\dagger}$ & 0.007$^{\dagger}$ & 0.064$^{\dagger}$ & 0.009$^{\dagger}$ \\\hline
\end{tabular}
}
\end{centering}


\caption{Evaluation results for affinity propagation and baseline clustering with additional features.
$^\dagger$ indicates significant improvement of a system with features in comparison to its featureless counterpart in table \ref{tab:eval-cl-tr}.
$^{*}$ indicates significant improvement of a system with path weight scoring over its counterpart without or vice versa.
$^{a}$ and $^b$ mark significant improvement of $TR^{(date)}$ system with path over $TR^{(temp)}$ with path and $TR^{(global)}$ with path respectively (or vice versa).
}
\label{tab:results-features}

\end{table}




\subsection{Token Constraints}

\begin{table}
\makebox[\textwidth][c]{

\cellspacetoplimit 3pt
\cellspacebottomlimit 3pt

\begin{tabular}{|Sl|Sl|SlSl|SlSl|SlSl|}

\hline

System & Date & \multicolumn{ 2 }{c|}{Concat} & \multicolumn{ 2 }{c|}{Agree} & \multicolumn{ 2 }{c|}{Align} \\
 & F1 & R1 & R2 & R1 & R2 & R1 & R2 \\\hline\hline

\multicolumn{8}{|c|}{ \textbf{Timeline 17} }\\\hline
\multicolumn{8}{|l|}{ \textit{Affinity Propagation} }\\\hline
\multirow{2}{*}{$TR^{(date)}$ + Feat. + Path}  & 0.518 & 0.361 & 0.088 & \textbf{0.091}    & \textbf{0.027}    & \textbf{0.100}    & 0.028 \\
    & \small \textcolor{red}{-0.001}    & \underline{\small \textcolor{green}{+0.027}}  & \underline{\small \textcolor{green}{+0.008}}  & \underline{\small \textcolor{green}{+0.012}}  & \small \textcolor{green}{+0.004}  & \underline{\small \textcolor{green}{+0.011}}  & \small \textcolor{green}{+0.003} \\\hline
\multirow{2}{*}{$TR^{(date)}$ + Feat.}  & \textbf{0.526}    & 0.346 & 0.069 & 0.070 & 0.018 & 0.078 & 0.019 \\
    & \underline{\small \textcolor{green}{+0.015}}  & \small \textcolor{red}{-0.009}    & \underline{\small \textcolor{red}{-0.008}}    & \underline{\small \textcolor{red}{-0.008}}    & \small \textcolor{red}{-0.002}    & \underline{\small \textcolor{red}{-0.010}}    & \small \textcolor{red}{-0.003} \\\hline

\multirow{2}{*}{$TR^{(global)}$ + Feat. + Path}   & 0.506 & \textbf{0.366}    & 0.089 & 0.084 & 0.022 & 0.095 & 0.024 \\
    & \small \textcolor{red}{-0.001}    & \underline{\small \textcolor{green}{+0.032}}  & \small \textcolor{green}{+0.005}  & \small \textcolor{green}{+0.009}  & 0.000 & \underline{\small \textcolor{green}{+0.011}}  & \small \textcolor{red}{-0.001} \\\hline
\multirow{2}{*}{$TR^{(global)}$ + Feat.}   & 0.506 & 0.352 & 0.075 & 0.070 & 0.018 & 0.079 & 0.020 \\
    & \small \textcolor{red}{-0.002}    & \small \textcolor{red}{-0.004}    & \small \textcolor{red}{-0.002}    & \small \textcolor{red}{-0.005}    & \small \textcolor{red}{-0.001}    & \small \textcolor{red}{-0.006}    & 0.000 \\\hline

\multirow{2}{*}{$TR^{(temp)}$ + Feat. + Path} & 0.500 & 0.361 & \textbf{0.091}    & 0.086 & \textbf{0.027}    & 0.098 & \textbf{0.029} \\
    & \small \textcolor{red}{-0.007}    & \underline{\small \textcolor{green}{+0.028}}  & \small \textcolor{green}{+0.007}  & \small \textcolor{green}{+0.011}  & \small \textcolor{green}{+0.004}  & \underline{\small \textcolor{green}{+0.013}}  & \small \textcolor{green}{+0.004} \\\hline
\multirow{2}{*}{$TR^{(temp)}$ + Feat.} & 0.513 & 0.355 & 0.076 & 0.073 & 0.018 & 0.083 & 0.020 \\
    & \small \textcolor{green}{+0.002}  & \small \textcolor{red}{-0.007}    & \small \textcolor{red}{-0.003}    & \small \textcolor{red}{-0.001}    & 0.000 & \small \textcolor{red}{-0.002}    & 0.000 \\\hline\hline

\multicolumn{8}{|c|}{ \textbf{Crisis} }\\\hline
\multicolumn{8}{|l|}{ \textit{Affinity Propagation} }\\\hline
\multirow{2}{*}{$TR^{(date)}$ + Feat. + Path}  & 0.304 & \textbf{0.360}    & \textbf{0.080}    & \textbf{0.071}    & \textbf{0.026}    & \textbf{0.088}    & \textbf{0.028} \\
    & \small \textcolor{red}{-0.014}    & \underline{\small \textcolor{green}{+0.018}}  & \small \textcolor{red}{-0.001}    & \small \textcolor{red}{-0.001}    & \small \textcolor{red}{-0.001}    & \small \textcolor{green}{+0.001}  & \small \textcolor{red}{-0.001} \\\hline
\multirow{2}{*}{$TR^{(date)}$ + Feat.}  & 0.299 & 0.342 & 0.057 & 0.056 & 0.012 & 0.072 & 0.015 \\
    & \small \textcolor{red}{-0.011}    & \small \textcolor{red}{-0.007}    & \underline{\small \textcolor{red}{-0.003}}    & \small \textcolor{green}{+0.001}  & \small \textcolor{green}{+0.001}  & \small \textcolor{green}{+0.001}  & \small \textcolor{green}{+0.001} \\\hline

\multirow{2}{*}{$TR^{(global)}$ + Feat. + Path}   & 0.297 & 0.358 & 0.074 & 0.058 & 0.015 & 0.073 & 0.017 \\
    & \small \textcolor{red}{-0.012}    & \underline{\small \textcolor{green}{+0.020}}  & \small \textcolor{green}{+0.002}  & \small \textcolor{green}{+0.002}  & 0.000 & \small \textcolor{green}{+0.004}  & 0.000 \\\hline
\multirow{2}{*}{$TR^{(global)}$ + Feat.}   & 0.288 & 0.334 & 0.053 & 0.040 & 0.007 & 0.054 & 0.009 \\
    & \small \textcolor{red}{-0.005}    & \small \textcolor{red}{-0.003}    & \small \textcolor{red}{-0.004}    & \small \textcolor{red}{-0.005}    & \small \textcolor{red}{-0.002}    & \small \textcolor{red}{-0.006}    & \small \textcolor{red}{-0.002} \\\hline

\multirow{2}{*}{$TR^{(temp)}$ + Feat. + Path} & 0.304 & 0.351 & 0.074 & 0.059 & 0.015 & 0.074 & 0.017 \\
    & \small \textcolor{red}{-0.016}    & \underline{\small \textcolor{green}{+0.020}}  & \small \textcolor{green}{+0.003}  & 0.000 & \small \textcolor{red}{-0.001}    & \small \textcolor{green}{+0.003}  & 0.000 \\\hline
\multirow{2}{*}{$TR^{(temp)}$ + Feat.} & \textbf{0.316}    & 0.353 & 0.061 & 0.051 & 0.009 & 0.066 & 0.011 \\
    & \underline{\small \textcolor{green}{+0.021}}  & \small \textcolor{green}{+0.009}  & \small \textcolor{green}{+0.002}  & \small \textcolor{green}{+0.006}  & \small \textcolor{green}{+0.002}  & \small \textcolor{green}{+0.005}  & \small \textcolor{green}{+0.002} \\\hline

\end{tabular}
}

\caption{Results under token constraint. Smaller numbers indicate change of the score from the sentence constraint case. Underlined changes in score are statistically significant.}
\label{tab:results-token}
\end{table}

We now evaluate the effect of constraining the number of tokens, instead of the number of sentences, on our best systems. In this context, we also add the $f_{length}$ scoring function to the systems without path weight. While it made sense to remove it for the sentence constraint setting, in the token constraint setting the original reasoning of \citet{banerjee} applies that many short sentences are preferable to few long ones.

We evaluate the affinity propagation based systems with features under the token constraint in table \ref{tab:results-token} and compare them to the results for the sentence constraint.

We can find the same general picture in the results, as in the sentence constraint case. The $TR^{(date)}$ system with path weight remains the strongest system on the Crisis corpus and all systems with path weight are closely tied on the Timeline 17 dataset.
While the effect of switching to token constraint is mostly positive for the path weight scoring functions, sometimes even resulting in significant score increases, it is mostly negative for systems without the path constraint. We speculate that this positive effect can in part be explained by the fact that the path weighted system selects very short sentences in both settings.
For example, the average length of a sentence selected by the AP+$TR^{(temp)}$ setting is around 45 in the sentence constraint setting and around 40 in the token constraint setting. Meanwhile the system with $f_{\mathit{path}}$ selects sentences with an average length of 15 tokens in both settings.
By switching to the token constraint setting, the system with $f_{\mathit{path}}$ can integrate more of these short sentences in the summaries and thus improve its scores.


%We present the results for token constraints in table \ref{tab:results-token}.
%
%The results for most systems change only slightly for the worse and changes are not statistically significant.
%Interestingly this is not true for the system with path weight, which improves by up to $0.03$ in some metrics.
%We speculate that this is a consequence of the fact, that the path system is constructing very short sentences in the first place.
%While the average number of tokens of the system without path weight is $xx$ with sentence constraints, for path weight it is only $xx$.
%This means token constraints allow the path system to output more sentences per date than before.

\section{Discussion}

The evaluation has yielded a number of interesting insights about the performance of our system:

\begin{enumerate}
    \item{The choice of clustering algorithm is critical for performance. Our affinity propagation clustering method performs much better than the original document based clustering in all metrics with all features enabled.}
    \item{Including TLS inspired features generally leads to a huge improvement in performance over just using keyword scores.}
    \item{Variations in the TextRank based informativeness scores have some influence when used on their own, but are generally not as important as the aforementioned features, except for one outlier.}
    \item{Including path weight of generated sentences in the scoring function generally improves ROUGE-2 scores.}
\end{enumerate}

While the last finding may potentially also hold on traditional MDS corpora, the first and second point are more specific to TLS.
They support the idea that abstractive TLS is not just abstractive MDS on different points in time, but has some unique challenges.
Still, our system can achieve good results by building on abstractive MDS to create an abstractive TLS system.
This makes it seem worthwhile to investigate abstractive TLS as a problem related to abstractive MDS, but not equivalent to it.

Our evaluation on both sentence constraint and token constraint settings has shown, that our system is capable of performing well in both settings. It is also able to outperform an extractive baseline that is commonly used in the evaluation of abstractive TLS.
Unlike the baseline system, we do not require additional filtering of the corpus with a hand-crafted query to achieve our results.
Together with the results of our oracle summarization, which show that using abstractive summarization can push the potential scores of TLS, this emphasizes that using advances in abstractive MDS to create abstractive TLS systems is both worthwhile and feasible.

Besides these key points, the evaluation has also lead to two other interesting findings, that might be relevant to TLS as a whole.

The first finding is a potential problem with the Crisis dataset for evaluation.
We have identified one case where good performance on only one of four parts of the corpus has lead to disproportionate change in the total results.
This is probably a result of the relatively small variety of topics of the dataset. While it has many more reference timelines per topic than the Timeline 17 dataset and even more articles, there are only four sets of articles from which these timelines are all ultimately generated. We argue that this is undesirable if average ROUGE scores over all generated timelines are used as an evaluation metric, as this means there is considerable overlap in the documents that each timeline is generated from and thus timelines are not independent of each other.

The second interesting ancillary finding is the interaction between date selection and abstractive summarization.
Our experiments have shown that if we treat date selection and summarization as two completely separate problems, the date selection algorithm may choose dates that are difficult to summarize for the summarizer, resulting in lower ROUGE scores. This highlights that it can be important for the summarizer to be able to communicate with the date selection algorithm to balance date importance and the feasibility of creating summaries for these dates.

\chapter{Grammaticality Evaluation} \label{sec:results-manual}

Besides the automatic evaluation presented in section \ref{sec:results-auto}, we also conduct a manual evaluation of the grammaticality of our system's output.
Ideally, we would evaluate all our systems for both informativeness and readability. This is, however, not feasible given our limited resources.
We thus reduce the scope of our evaluation to only include the grammaticality of individual sentences.
To further reduce the size of the annotation task, we select a small subset of sentences from each timeline.
We then present these samples to an annotator, who is asked to rate the grammaticality of each sentence on a numeric scale.

\section{Setup}

We follow the evaluation procedure for grammaticality devised by \citet{mckeown} that was also used by \citet{filippova}.
Annotators were asked to annotate each sentence on a scale from $1-3$. \footnote{In the original implementation it is $0-2$. However, using this scale was not technically possible on our annotation platform.} They were asked to rate a sentence with 3, if it did not contain any grammatical error. A score of 2 was given to sentences with at most one small error, that can be fixed by small editing. Small errors include incorrect articles or errors in agreement. In all other cases a score of 1 should be assigned.

%We select three systems for evaluation. Besides the baseline system, we include our two best performing systems, the AP+$f_{ ling*TR^{(date)}*\mathit{dateref}*\mathit{csize}*\mathit{path} }$ system and the AP+$f_{ ling*TR^{(date)}*\mathit{dateref}*\mathit{csize} }$ system.
We select three systems for evaluation. Besides the MDS baseline system (with ILP), we include the AP+$TR^{(temp)}$ + Feat. + Path. system, along with its counterpart without path weight. We refer to the former as AP+path and to the latter as AP for brevity. While it would be interesting to see the performance of the other informativeness measures, we do not have the resources to conduct a full investigation into all top systems.
We do, however, speculate that the results would be similar for all three systems, as they differ only in the scoring of content not of linguistic quality.
To keep the scope of the evaluation small, we only evaluate timelines generated under sentence count constraint, which is the default setting for timeline summarization. We also sampled only a couple of sentences from each timeline, to further reduce the size of the corpus.

We sampled five sentences from each individual timeline our system has generated.
The sentences were then split into three equally-sized batches. Each batch was annotated three times by three different annotators.
Annotators were not aware that sentences were generated by different systems and the sentences were presented in random order, to prevent potential bias.

We have recruited a total of nine annotators. All of them are proficient in English, but they are not native speakers.
Each annotator was given one batch to annotate along with instructions on the rating scale. We ran a small pilot study with only four annotators and 20 items. We asked annotators to report any problems with the instructions. We found that it was unclear to annotators, whether headline style phrases are permissible.
As some of the reference timelines were also written in that style, we explicitly allowed that in our guidelines.

The score of each item is determined by majority voting. If there is a tie, the lower of the tied ratings is chosen.

\section{Results and Analysis}

We use Krippendorff's Alpha \citep{krippendorff} to compute the reliability of our annotators.
The reliability is very low at only $0.316$. We think that this is due to the fact that none of our annotators are native speakers.

We show both the average score of each system and the percentage of times it received a particular rating in table \ref{tab:results-manual}.
We include the latter following the suggestion of \citet{filippova}, as practically speaking the scores $3$ and $2$ are much closer to each other than $2$ and $1$, as 2 indicates only minor errors, while 1 also includes cases in which entire parts of the sentence may be missing.
%The table also includes the results \citet{filippova} repr

The results show that none of the systems receives consistently good ratings. The majority of sentences for baseline and AP system receive the worst score of 1, indicating mayor errors, except for the AP+path system.
The best results are achieved by the AP+path system, which receives the best score of 3 in one-third of all cases. The improvement seems to be mostly due to the $f_{path}$ component of the scoring function, as the AP performs much worse, with more than half of all sentences receiving a score of 1.
We investigate a potential reason for this in section \ref{sec:errors-grammar}.

Surprisingly, even the baseline system outperforms AP. We speculate that this is for two reasons:

\begin{itemize}
    \item{The $f_{ling}$ score has proportionally larger influence, because it has to compete with fewer other scores.}
    \item{As shown in table \ref{tab:genstats}, the clusters of the document based clustering method are smaller than those produced by AP. Smaller clusters produce graphs with less alternatives and thus reduce the chance to improperly combine paths from different sentences.}
\end{itemize}

When we compare the results of our system to the evaluation of \citet{banerjee} and \citet{filippova}, we notice that our results are lower than we would expect given the baseline systems. \citet{banerjee} let human annotators rate linguistic quality of whole summaries on a scale from 1 to 5 and report an average score of $3.63$, while human summaries receive an average of $4.35$ points.
While this score is not necessarily comparable to our rating scale and setting, \citet{filippova} reports an average score of $1.44$ with the same rating scale we use in their evaluation of headline quality. Note that her rating scale starts from $0$, not $1$, so this is equivalent to a score of $2.44$ in our setting. This score is also much better than what we achieve, although we should note that she conducts headline summarization, not full MDS.

Still, these results suggest that the core problem might not be in summarization component itself, but in the way we cluster, score and select sentences.

Overall the results, even while unreliable, indicate that the grammaticality of our system output is not satisfactory and further work is required to improve the readability of our system output.

\begin{table}

\centering\begin{tabular}{|l|c|ccc|}
\hline
\textbf{System} & \textbf{Avg.} & \textbf{Score 1} & \textbf{Score 2} & \textbf{Score 3} \\\hline
AP+Path &  2.04 & 0.38 & 0.195 & 0.425 \\
AP & 1.98 & 0.412 & 0.196 & 0.392 \\
Baseline & 1.5 & 0.668 & 0.166 & 0.166 \\\hline
\end{tabular}


\caption{Average grammaticality score and relative frequency of scores for the three systems we evaluated.}
\label{tab:results-manual}
\end{table}

\chapter{Error Analysis and Examples}  \label{sec:errors}

Sections \ref{sec:results-auto} and \ref{sec:results-manual} give us important data to gouge the performance of our system, but it is also interesting to look at the full timelines.
While we lack the resources to do an in-depth evaluation of all timelines, we have identified several problems manually.

\section{An Example Timeline}

\begin{table}
\begin{tabularx}{1.2\textwidth}{|X|X|}
\hline
\textbf{2010-04-20} & \textbf{2010-04-20} \\\hline
Explosion and fire on the BP-licensed Transocean drilling rig Deepwater Horizon in the Gulf of Mexico . \newline Eleven people are reported missing and approximately 17 injured. \newline A blowout preventer, intended to prevent release of crude oil, failed to activate. \newline Deepwater Horizon oil rig fire leaves 11 missing. & The spill began with the april 20 oil rig killed 11 workers. \newline BP rig Deepwater Horizon oil spill began on 20 , killing 11 workers. \\\hline
\textbf{2010-04-22} & \textbf{2010-04-22} \\\hline
Deepwater Horizon rig sinks in 5,000 ft of water. \newline Reports of a five-mile-long oil slick. \newline Search-and-rescue operations by the US National Response Team begin . & The oil spill loomed, officials said (reuters) . \newline But BP , sank on 22 , two days after an explosion killed 11 workers . \\\hline
\textbf{2010-04-23} & \textbf{2010-04-23} \\\hline
The US coast guard suspends the search for missing workers, who are all presumed dead. \newline The rig is found upside down about a quarter-mile from the blowout preventer. \newline A homeland security department risk analysis says the incident ''poses a negligible risk to regional oil supply markets and will not cause significant national economic impacts''. \newline White House press secretary Robert Gibbs says: ''I doubt this is the first accident that has happened and I doubt it will be the last.'' \newline Deepwater Horizon clean-up workers fight to prevent disaster & BP plc said friday it has activated an extensive oil spill. \newline Coast Guard said friday, though officials were trying to contain a massive oil is leaking from a rig that sank off louisiana's coast's fragile ecosystem. \\\hline
\textbf{2010-04-24} & \textbf{2010-04-25} \\\hline
Oil is found to be leaking from the well. \newline A homeland security report on critical infrastructure says the problem has ''no near-term impact to regional or national crude oil or natural gas supplies.'' & It is a big change from yesterday ... this is a very serious oil spill, absolutely, said the leak was a new discovery but could have begun when the rig sank on thursday, two days after the initial explosion. \newline High tides and waves through sunday could push oil deep into the inlets, ponds and lakes in south-east louisiana. \\\hline
\textbf{2010-04-25} & \textbf{2010-04-27} \\\hline
US coast guard remote underwater cameras report the well is leaking 1,000 barrels of crude oil per day (bpd). \newline It approves a plan for remote underwater vehicles to activate a blowout preventer and stop the leak. \newline Underwater robots try to seal well & BP said it would begin drilling a relief well to redirect the leaking oil well. \newline April 27: weathered oil rig is seen in the Gulf of Mexico near Louisiana . \\\hline
%\textbf{2010-04-30} & \textbf{2010-05-02} \\\hline
%in the 1989 exxon valdez accident , as environmentalists are now claiming , and it takes an additional 60 to 90 days to drill a relief well and stop the leak , the total spill could be about 1 million barrels of crude oil were spilled into prince william sound , alaska , cleanup costs were an estimated \$ 2 billion , fitch said . \newline the oil well continues to leak an estimated 200,000 gallons of crude oil spill could become the nation 's worst environmental disaster in decades . & it could be capturing up to 85 percent of the oil . \newline the company began drilling the first of two relief wells . \\\hline
\end{tabularx}
\caption{Example Timelines. The left timeline was created by a human author, the right was generated by the AP+$f_{TR^{(temp)} * dateref * csize * path}$ system. Capitalization and punctuation edited for clarity.}
\label{tab:tlbp}
\end{table}



%\begin{table}
%\begin{tabularx}{\textwidth}{|X|X|}
%\textbf{2012-01-22} & \textbf{2012-01-22} \\\hline
%the un estimates more than 5,400 people have died since protests broke out in march 2011 . & the un estimates more than 5,400 people have been killed since protests %broke out in march 2011 . \\\hline
%\textbf{2012-02-04} & \textbf{2012-02-04} \\\hline
%russia and china vetoed a draft resolution that backed an arab plan to facilitate political transition in syria . \newline activists say hundreds of people killed %since feb. 4 . & last month vetoed a draft resolution that backed an arab league plan to facilitate political transition in syria . \newline a u.n. resolution this %month calling on president bashar al - assad to cede power . \\\hline
%\textbf{2012-02-26} & \textbf{2012-02-20} \\\hline
%Syrian president bashar assad casts his ballot next to his wife during a referendum on the new constitution , in damascus , syria , on sunday feb. 26 , 2012 . \newline% on sunday , Syrians€œ voted in a constitutional referendum in polling centers across the country . & opposition activists said five people killed in government %shelling of baba amro district of homs began on february 3 . \newline the assad government says more than 2,000 soldiers and security forces have killed since , %according to human rights groups . \\\hline
%\end{tabularx}
%\end{table}

Table \ref{tab:tlbp} shows the first five dates of a timeline generated on the BP-oil spill disaster from the Timeline 17 dataset by the AP+$f_{ ling*TR^{(temp)}*\mathit{dateref}*\mathit{csize}*\mathit{path} }$ system with sentence constraint.
For comparison, we also show the first five entries of the human-authored timeline.

We can see that the generated timeline covers some of the same general events as the human timeline.
The explosion of April 20th, the sinking of the rig on the 22nd and the beginning of the oil spill on April 24th, although our system misses the date by one day. The human-authored timeline, however, is far more detailed in many aspects and also includes many points the system does not.
Another notable difference is the different length of summaries in both timelines. This is a result of the way constraints are determined. While the human timeline has an average length of two sentences, the entries for the first few days are up to four sentences long, while later entries are shorter.
As we have a per-date-length constraint, our system can not put additional focus on some dates at the expense of others. This suggests that it might be worthwhile to change constraints to allow for this behavior in the future.

Besides this, there are a number of additional problems that are demonstrated in this example.

\subsubsection{Corpus Artifacts}

As we can see in the summary for the 22nd, our system sometimes includes artifacts from the original articles, in this case the news agency ''Reuters''. Our system is not natively capable of identifying such artifacts and may even score their presence positively, if they occur frequently.
''Reuters'' for example appears a total of 1444 times in the BP oil spill part of the Timeline 17 corpus.

A similar problem can be found on April 25th, where we have a line that seems to describe an image, which is naturally not present in the timeline.

While it would of course be preferable, if we filtered out such cases directly in the system, an alternative fix for this problem would be by cleaning the original corpus. This work, however, is outside of the scope of this project.

\subsubsection{Redundancy}

We can see the system has some problems with redundancy in the first summary. We have found that this system is aggravated when more sentences are allowed per date, as can be seen from the following summary for the first date by the same system with token constraint:

\countquotenosource{
bp rig deepwater horizon oil spill began on 20 , killing 11 workers .
oil has been leaking into the gulf since the rig exploded and sank last month , killing 11 workers .
the spill began with the april 20 oil rig killed 11 workers .
the deepwater horizon oil rig exploded on 20 .
\label{ex:redundancy}
}

Here, the same fact is stated in slight variation multiple times.

This points to multiple problems:

\begin{itemize}
    \item{The clustering algorithm does not properly group all sentences describing the same event in one cluster, as only one candidate is included per cluster }
    \item{The redundancy component of the submodular function does not prevent the inclusion of the redundant sentences }
\end{itemize}

The first concern suggests it might be worth investigating the output of the clusterer more closely.
We refer to the clusters as cluster 1, 2, 3 and 4, corresponding to the order of their representatives in example \ref{ex:redundancy}.
We find that the four clusters differ in subtle ways.
For example, ~70\% of sentences in cluster 3 contain the word ''explosion'', while the rest of the clusters only contain it once. They instead contain the word ''exploded'' in a similar frequency.
While the word ''spill'' is mentioned in clusters 1 and 3 in almost one-third of the sentences, it is only once mentioned in clusters 2 and 4, which have a higher frequency of the word ''leaking'' instead, as in ''leaking oil''.
It is unsurprising that our tf-idf model is not able to pick up on these slight variations that allow the sentences to end up in different clusters. As \citet{chieu} have already noted, cosine similarity does not reliably predict whether two sentences report the same event.

The problem is further aggravated by the fact that the initial explosion of the platform is a key event in the oil spill corpus and is thus mentioned relatively frequently. The clusters the sentences were generated from are the 3rd, 7th, 2nd and 1st largest clusters in the corpus. Members of these clusters thus have a large number of very similar sentences. The preference values that guide how likely it is for a sentence $s$ to become an exemplar are based on the median of incoming similarities for $s$ and thus these values are bound to be higher for these sentences.
Higher preference values in turn result in more, smaller clusters \citep{ap}, which further explains the results we see here.

Fixing this problem would require us to introduce a more sophisticate similarity metric. Alternatively, it might be reasonable to investigate ways in which we can avoid the bias towards splitting these large clusters of sentences up into smaller sentences.

For the second problem, we have formed two hypothesis for the cause. Either the scores of the sentences are so high, they are still attractive, even if other sentences from the cluster have been selected already, or the partitioning does not assign the sentences into the same blocks.
An investigation of the partitioning for the redundancy function reveals the latter is true. None of the four sentences were clustered into the same block. Thus the k-means algorithm divides the huge number of candidates that were generated from the large clusters we have described before into multiple partitions, so they are not recognized as being redundant.

While this observation might point to problems in the selection of the number of blocks in the partition, we speculate that there is another factor at work here.
Generally, k-means is not good at detecting clusters which have different sizes and densities \citep{introduction-data-mining-tan}.
This is a problem in our case, as very important events that have multiple (sentence generation) clusters that describe them, as shown above, will lead to a huge number of very similar sentences being generated for them. To illustrate that point, consider that the most productive of the four clusters mentioned above has 672 distinct candidate sentences after filtering, while some cluster generate only one candidate.
Due to the limitations of k-means, these large clusters of similar sentences may be broken up into smaller clusters, thus resulting in redundant sentences ending up in different blocks of the partitioning.

To alleviate this problem we would need to either find a different clustering algorithm for our submodular function, or an alternative way to formulate redundancy that does not rely on clustering.

\section{Ungrammatical Sentences} \label{sec:errors-grammar}

Besides the problems with our timelines in general, mentioned in the previous section, we also find some problems with the linguistic quality of selected sentences, which we outline here.

\subsubsection{Overly Long Sentences}

One problem we have encountered when we drop the normalization for length in the sentence constraint setting is that the system tends to join multiple sentences together to maximize keyword score:

\countquotenosource{
More than half of the dead were reported in cairo overnight anger : 26 people have been injured in the protests , which have included violent clashes during the 18-day uprising in january revolution that toppled hosni mubarak today sacked his entire cabinet and promised reforms after the violent protests running battles were still taking place this morning despite a curfew and extra security forces stormed a sit - in at tahrir staged by several hundred protesters were killed in four days.
}

While this might partially be fixed by introducing a sentence length constraint, it points to a deeper problem, where the language model is not adequately able to score this sentence as ungrammatical.
This mirrors a problem encountered by \citet{mckeown}, where the language model score they use to assess grammaticality of generated sentences also did not always score grammatical sentences higher than ungrammatical ones. In fact it is not surprising this problem arises with this particular instance. The sentence is a concatenation of multiple grammatical sentences, where the problems are mostly at the boundaries, where content from different sentences is fused together. As the language model only sees one trigram at a time, this is difficult to detect, as the most parts of the sentence appear normal in isolation.

\subsubsection{Incomplete Sentences}

On the other end of the spectrum, we have sentences where crucial parts are missing, making the sentence ungrammatical. It thus appears that the heuristic filtering methods we have introduced for sentences are not sufficient.
See the following example, generated by the AP+$f_{ ling*TR^{(date)}*\mathit{dateref}*\mathit{csize}*\mathit{path} }$ where we miss a verb in the subclause.

\countquotenosource{
    U.S. defense robert gates said the fall of saleh as a ''real problem.''
}

While our constraints during sentence generation enforce that there is a verb present in each generated sentence, they are not aware of potential subclauses, that also require verbs.

To alleviate both of the above mentioned problems, future work should focus on finding ways to score the linguistic quality of sentences taking into account their syntactic structure.

\section{Factually Wrong Sentences}

Extractive summarization approaches have the advantage that all sentences that they include in their documents have been found in their source documents. As long as their source documents are reliable, the sentences are thus most likely not completely incorrect, although they might be missing context.
This guarantee, however, can not be made for our abstractive system, which we demonstrate here.

Take the following sentence from a timeline generated by our system on the Michael Jackson timeline:

\countquote{
    Judge Michael Pastor said Murray had shown no remorse for Jackson's death.
}{Capitalization edited for clarity}

While superficially there is nothing wrong with this sentence, a look at the original documents reveals the following original sentence, from which the statement was apparently generated:

\countquote{
In a sentencing memorandum delivered to Judge Michael Pastor in advance of Tuesday 's hearing , prosecuting lawyer David Walgren said Murray had shown no remorse for Jackson 's death .
}{Source: Timline 17 corpus, document mj/2011-11-29/130.htm.txt}

There is a significant difference between the statement being made by the sentencing judge and a prosecuting lawyer, which our system is unsurprisingly not aware of.
While we do not have the means to widely fact-check our system output, this example shows that not everything our system outputs is necessarily correct.
This would be a serious problem if a system such as ours would actually be relied upon to summarize news events. The fact that the sentence itself has no tell-tale signs of erroneous composition would make it impossible for anyone reading the sentence to deduce its incorrectness without looking up the story itself.

While we might be able to fix this particular problem by restricting permissible paths and improving clustering, it also shows that there is an inherent risk to systems which only work on sentences superficially, instead of trying to extract their underlying meaning.
There is no guarantee that small transformations do not distort the meaning of sentences and thus the understanding the reader of a timeline develops of the events that have transpired.


%\section{Problems with Current Evaluation Metrics}

\chapter{Conclusion and Future Work}  \label{sec:conclusion}

%\section{Conclusion}

We have presented a system for abstractive timeline summarization by capitalizing on an existing system for MDS and shown that it performs well on two common TLS corpora. To our knowledge, this is the first investigation into the possibility of abstractive TLS.

It has become clear that using an abstractive MDS system for TLS is at least in this case not just a matter of running the system on a different corpus but requires us to make huge modifications to achieve satisfactory results. This mirrors the results of \citet{markert}, who have shown that adapting an extractive MDS system to TLS can profit from introducing timeline specific features, although our results are more extreme in the effect of these modifications.

Our experiments show that while we can use an abstractive summarization system for abstractive TLS, much better results can be achieved by tailoring it for TLS.
We have suggested extensive modifications to almost all parts of the system, which allow it to take into account the specific challenges of TLS and also scale up to the large corpora typical to that task.

%Experimental evaluation has shown that our modifications allow the system to perform much better than the unaltered baseline MDS system. %This mirrors results the results of \citet{yan-trans}, who show that 
By using the submodular TLS framework proposed by \citet{markert} to allow for global optimization of the timeline summaries and adapting the clustering component to TLS, we can achieve significant improvements over the abstractive MDS baseline system.
By adding features motivated by extractive TLS and MDS approaches we can further improve our system to outperform an extractive baseline.
By including information from the sentence generation system, we are able to both improve grammaticality of our system output and ROUGE scores.

There are, however, still significant challenges for the system, most notably the poor linguistic quality of the timelines revealed by human annotators.
Future work should try to address this issue by incorporating better filtering of candidates and finding better ways to score linguistic quality of sentences.
Another challenge to our system is the scale of the summarization problem. While extractive TLS typically works on sentences that may number in the ten to hundred thousands, selecting from a pool of abstractive candidate sentences can grow the size of the problem a ten fold. This makes it challenging to scale existing solutions, which in our case reflects on the ability of our system to avoid redundancy.
Future work should also focus on addressing these problems in scalability.

More generally, a comparison of abstractive oracle summarization shows that using abstractive summarization can push the possible scores of a system beyond what is achievable with extractive systems. This advantage, in combination with the observation that abstractive TLS is not simply abstractive MDS on larger corpora, but a problem in its own right, makes it seem worthwhile to further investigate abstractive MDS systems and try to find ways to apply them to TLS.

\bibliographystyle{plainnat}
\bibliography{bib} 


\begin{appendices}
\chapter{Proof of Performance Guarantees}

In the following section we adapt the proof of \citet{markert} to the constraints we have introduced in section \ref{sec:system-selection}.

We first state a couple of the definitions we require for the proof.

\begin{definition}
A pair $(X, \mathcal{I})$, where $X$ is a finite set and $\mathcal{I} \subset 2^X$ is a collection of subsets of $X$, is an independence system, if the following holds: (i) $\emptyset \in \mathcal{I}$ (ii) $B \in \mathcal{I}$ and $A \subseteq B$ implies $A \in \mathcal{I}$
\end{definition}

\begin{definition}
Let $(X, \mathcal{I})$ be an independence system. The set of bases for a set $Y \subseteq X$ is $\mathcal{B}(Y) = \{B | B \subseteq Y, B \in \mathcal{I} \mathit{ and } B \mathit{ is maximal} \}$
\end{definition}


\begin{definition}
A tuple $(X, \mathcal{I})$ is a $p$-independence system, if for all $Y \subseteq X$, we have that
\begin{equation}
 \frac{\max_{A \in \mathcal{B}(Y)} |A|}{\min_{A \in \mathcal{B}(Y)} |A|} \leq p
\end{equation}
\end{definition}

Following \citet{markert}, we know that if we can proof that the set of generated sentences $G$ and our constraints form a $p$-independence system $(G, \mathcal{I})$, we can guarantee that our system finds a summary with a score that is within a constant factor of $1 / (1 + p)$ of the optimal score.

\section{Proof under Cluster Membership Constraint} \label{sec:proof-cluster-constr}

We first prove that the constraints \ref{eq:constr-cluster}, \ref{eq:constr-date} and \ref{eq:constr-len-sent} form a $p$-independence system with $p = k_{sent}$.
The proof is almost the same as the one \citet{markert} give for their constraints, but includes the clustering constraint \ref{eq:constr-cluster}.

Let $Y \subseteq G$ and let $m = \min \{ l, |\{\mathit{date}(s) | s \in Y\}| \}$.
We show that 
\begin{equation}
\max_{A \in \mathcal{B}(Y)} |A| \leq m * k_{sent} \label{eq:upper}
\end{equation}
 and

\begin{equation}
\min_{A \in \mathcal{B}(Y)} |A| \geq k_{sent} \label{eq:lower}
\end{equation}
and thus 
\begin{equation}
\frac{\max_{A \in \mathcal{B}(Y)} |A|}{\min_{A \in \mathcal{B}(Y)} |A|}
\leq \frac{m * k_{sent}}{m} = k_{sent}    
\end{equation}

\subsection{Upper Bound} \label{sec:cluster-upper-bound}

We first prove that equation \ref{eq:upper} holds for all $Y \subseteq X$.

Let $A \in \mathcal{B}(Y)$ be a base of $Y$. We define the same equivalence relation $\sim_d$ as \citeauthor{markert} on $A \times A$:
$a \sim_d a'$ iff $\mathit{date}(a) = \mathit{date}(a')$.

We then have:
\begin{align}
|A| = \sum_{[a] \in A/\sim_d} |[a]| \\
    = \sum_{[a] \in A/\sim_d} |\{ a' \in A | d(a') = d(a) \}| \label{eq:A2} \\
    \leq \sum_{[a] \in A/\sim_d} k_{sent} \label{eq:A3} \\
    = m * k_{sent} \label{eq:A4}
\end{align}

The inequality \ref{eq:A3} follows directly from constraint \ref{eq:constr-len-sent}.
The equality \ref{eq:A4} follows from constraint \ref{eq:constr-date}.


\subsection{Lower Bound} \label{sec:proof-date-lower}

We prove that equation \ref{eq:lower} holds for all $Y \subseteq X$ via proof by contradiction.
Assume that there exists $A \in \mathcal{B}(Y)$ with $|A| < m$.
Since $m = \min \{ l, |\{\mathit{date}(s) | s \in Y\}| \}$ there exists at least one $s \in Y$ with $\mathit{date}(s) \not\in \{\mathit{date}(s') | s' \in A\}$.
Consider $A' = A \cup \{ s \}$. We show that $A' \in \mathcal{I}$, which is a contradiction to the maximality of $A$. \citet{markert} have already proven that $A'$ still conforms to the length constraints \ref{eq:constr-date} and \ref{eq:constr-len-sent}.
We thus only need to show that $A'$ also fulfills the cluster constraint \ref{eq:constr-cluster}.
As every sentence generated from the same cluster has the same date tag, we know it holds that
\begin{equation}
\forall s' \in \mathit{cluster}(s): \mathit{date}(s') \not\in \{\mathit{date}(s'') | s'' \in A \}
\end{equation}
and thus

\begin{equation}
\forall s' \in \mathit{cluster}(s) : s' \not\in A
\end{equation},

where $\mathit{cluster}(s)$ returns the cluster that $s$ was generated from.
It follows that $|\mathit{cluster}(s) \cap A'| = |\mathit{cluster}(s) \cap (A \cup \{ s \})| = |\{s\}| \leq 1$ and thus constraint \ref{eq:constr-cluster} holds.


\section{Proof under Token Length Constraint} \label{sec:proof-token-constr}

We prove that the constraints \ref{eq:constr-cluster}, \ref{eq:constr-date} and \ref{eq:constr-len-tok} form a $p$-independence system with $p = \frac{1}{\lfloor k_{tok} / h \rfloor}$.

Let $Y \subseteq G$ and let $m = \min \{ l, |\{\mathit{date}(s) | s \in Y\}| \}$.

We show that for a minimum sentence length $h$, we have 
\begin{equation}
\max_{A \in \mathcal{B}(Y)} |A| \leq m * \lfloor k_{tok} / h \rfloor
\end{equation}
and
\begin{equation}
    \min_{A \in \mathcal{B}(Y)} |A| \geq m
\end{equation}
and thus
\begin{equation}
\frac{\max_{A \in \mathcal{B}(Y)} |A|}{\min_{A \in \mathcal{B}(Y)} |A|}
\leq \frac{m * \lfloor k_{tok} / h \rfloor}{m} = \lfloor k_{tok} / h \rfloor
\end{equation}

The proof for the lower bound is the same as stated in section \ref{sec:proof-date-lower}, so we only need to proof the upper bound.

\subsection{Upper Bound}

We have proven in \ref{sec:cluster-upper-bound} that for every base $A \in \mathcal{B}(Y)$ for a set $Y \subseteq X$ we have

\begin{align}
|A| = \sum_{[a] \in A/\sim_d} |\{ a' \in A | d(a') = d(a) \}|
\end{align}

Due to equation \ref{eq:constr-len-tok}, we have
\begin{equation}
\sum_{ a' \in A | \text{date}(a') = \text{date}(a)} \mathit{length}(a') \leq k_{tok}
\end{equation}
for every $a \in A$, where $\mathit{length}(a) : X \rightarrow \mathbb{N}^+$ is the length of sentence $a'$ in tokens.

As $h$ is the minimum length of a sentence, we know that $\mathit{length}(a) \geq h$ and thus

\begin{equation}
\sum_{ a' \in A | \text{date}(a') = \text{date}(a)} h \leq k_{tok} \label{eq:sum-h}
\end{equation}

By transforming equation \ref{eq:sum-h}, we write
\begin{equation}
|\{ a' \in A | \text{date}(a') = \text{date}(a)\}| \leq k_{tok}/h
\end{equation}

As the cardinality of a finite set is a natural number, we can further bound the result by flooring the division:

\begin{equation}
|\{ a' \in A | \text{date}(a') = \text{date}(a)\}| \leq \lfloor k_{tok}/h \rfloor
\end{equation}


%As each sentence has a minimum length of $h$, there can be at most $\lfloor n / h \rfloor $ elements in $\{ a' \in A | d(a') = d(a) \}$.
Combining this with equation \ref{eq:A2}, we can now write

\begin{equation}
|A| \leq \sum_{[a] \in A/\sim_d} \lfloor k_{tok}/h \rfloor
\end{equation}

The rest of the proof is then analogous to \ref{sec:cluster-upper-bound}.



%
%Let $Y \subseteq G$. We need to prove that for some maximum token count per date $n \in \mathbb{N}^+%$ and a minmum number of tokens per sentence $h \in \mathbb{N}^+, h < n$, we have
%
%\begin{equation}
%\frac{\max_{A \in B(Y)} |A|}{\min_{A \in B(Y)} |A|} \leq \lfloor n/h \rfloor
%\end{equation}
%
%We can adapt the proof of \citet{markert} to a token constraint as follows.
%
%The lower bound is only dependent on the number of allowed dates $l$, not the number of sentences, %so it remains at $m = \min(l, |\{d(s) | s \in Y \}|)$.
%
%We now show how to change the prove of the upper bound for a token constraint.
%Let $A \in \mathcal{B}(Y)$.
%We define the same equivalence relation $\sim_d$ as \citeauthor{markert} on $A \times A$:
%$a \sim_d a'$ iff $d(a) = d(a')$.
%We then have:
%
%\begin{align}
%|A| = \sum_{[a] \in A/\sim_d} |[a]| \\
%    = \sum_{[a] \in A/\sim_d} |\{ a' \in A | d(a') = d(a) \}| \label{eq:A2}
%\end{align}
%
%We define $\mathit{len}(a) : A \rightarrow \mathbb{N}^+$ as the length of element $a$ in tokens.
%The length constraint that must be fulfilled for all $a \in A$ is:
%
%\begin{align}
%\sum_{a' \in N_a} \mathit{len}(a') \leq n \\
%N_a = \{ a' \in A | d(a') = d(a) \}
%\end{align}
%
%As each sentence has a minimum length of $h$, there can be at most $\lfloor n / h \rfloor $ %elements in $N_a$.
%Combining this with equation \ref{eq:A2}, we write:
%
%\begin{align}
%|A| \leq \sum_{[a] \in A/\sim_d} \lfloor n/h \rfloor
%\end{align}
%
%As the limit on the number of dates ensures that there are at most $m$ equivalence classes, we have %$|A| \leq \lfloor n/h \rfloor * m$.
%
%Combining this with the lower bound of $|A| \geq m$, we have:
%
%\begin{align}
%\frac{\max_{A \in B(Y)} |A|}{\min_{A \in B(Y)} |A|} = \frac{\lfloor n/h \rfloor * m}{m} \\
%= \lfloor n/h \rfloor
%\end{align}



%{a′ ∈ A | d(a′) = d(a)}

%We consider
%the equivalence relation ∼d defined on A × A by 581 a ∼d a′ if and only if d(a) = d(a′). This equiv- 582 alence relation induces a partition of A into its 583 equivalence %classes according to ∼d. We therefore 584
%have


\end{appendices}


\end{document}
